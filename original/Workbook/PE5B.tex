%!TEX root =  A_WS.tex

\section*{Practice Exam 5B}  
\markright{Practice Exam 5B}  
\addcontentsline{toc}{section}{Practice Exam 5B}

\emph{Try taking this version of the practice exam under testing conditions:  no book, no notes, no classmate's help, no electronics (computer, cell phone, television). Give yourself one hour to work and wait until you have tried your best on all of the problems before checking any answers.}

\noindent \hrulefill

\begin{enumerate}

\item The number of school children in the district whose first language in not English has been on the rise.  The equation describing the situation is $$C=673(1.043)^Y$$ where $C$ is the number of school children in the district whose first language is not English, and $Y$ is the number of years (from now).
\begin{enumerate}
\item Make a table showing the number of school children in the district whose first language is not English now, in one year, in two years, and in ten years. \emph{Don't forget now too.} 
\vfill
\vfill
\item What percent increase is implicit in this equation?
\vfill
\item Use successive approximation to determine when there will be over 1,700 school children in the district whose first language is not English Display your work in a table.  Round your answer to the nearest year. 
\vfill
\vfill
\item Show how to solve the equation to calculate when there will be over 1,700 school children in the district whose first language is not English. 
\vfill
\vfill
\end{enumerate}  

\newpage

\item The lottery jackpot started at \$600,000.  After 17 days the jackpot had increased to \$2.1 million.  The lottery is designed so that the jackpot grows exponentially.

\begin{enumerate}
\item Name the variables including units. 
\vfill
\item Write an equation describing the jackpot.  \emph{Hint:  find the daily growth factor.} 
\vfill
\vfill
\item By what percentage does the jackpot increase each day? 
\vfill
\item What will the jackpot be after 20 more days (after 37 days total)? 
\vfill
\end{enumerate} 

\newpage

\item The creeping vine is taking over Fiona's front lawn.  Write $V$ for the area covered by the vine (in square feet) and $Y$ for the years since she moved into her house.
\begin{enumerate}
\item When Fiona moved in, vine covered about 3 square feet.  She believes it has doubled each year since.  Write an exponential equation showing how the area covered by the vine is a function of time.  \emph{Stuck?  Try making a table first.} 
\vfill

\item At some point the vine will take over the entire lawn, so perhaps a saturation model would be better.  That equation might be $$V = 170 - 167 \ast .8^Y$$
Another equation would be a logistic model.  Perhaps $$V = \frac{129}{1+ 42 \ast .34^Y}$$
Fill in the corresponding rows of the table for each model.
\bigskip
\begin{center}
\begin{tabular} {|c|c |c|c|c|c|c|c|}\hline 
& & & & & & & \\
years &~ \quad 0 \quad~ &~ \quad 1 \quad~ &~ \quad 2 \quad~ &~ \quad 3 \quad~ &~ \quad 4 \quad~ &~ \quad 5 \quad~  &~ \quad 6 \quad~\\ 
& & & & & & &\\ \hline
& & & & & & & \\
area & & & & & & & \\  
exponential & & & & & & &\\ \hline 
& & & & & & & \\
area & & & & & & & \\  
saturation & & & & & & &\\ \hline 
& & & & & & & \\
area & & & & & & & \\  
logistic & & & & & & &\\ \hline
\end{tabular}
\end{center}
\vfill

\newpage
\hspace{-.5in} \emph{The problem continues \ldots}

\item Draw a graph showing all three models on the same set of axes.  

\bigskip
\begin{center}
\scalebox {1.0} {\includegraphics [width = 6in] {GraphPaper.jpg}}
\end{center} 
\bigskip

\end{enumerate} 

\newpage

\item Many different agencies are working to lower infant mortality.  Infant mortality is measured in deaths per thousand births.  The world infant mortality rate in 1955 was around 52 (per thousand births).  By the year 2000, it was down to around 23.

\hfill \begin{footnotesize}  Source: Wikipedia (Infant Mortality) \end{footnotesize}
\begin{enumerate}
\item Name the variables. 
\vfill
\item Write a linear equation modeling infant mortality. 
\vfill
\vfill
\item Now write an exponential equation modeling infant mortality. 
\vfill
\vfill

\item Compare the models projections for 1955, 1970, 1990, 2000, 2010, and 2020.  Summarize your findings in a table. 
\vfill
\vfill
\item The actual rates were 40 deaths per thousand births in 1970 and 28 deaths per thousand births in 1990.  Which model fits this additional data better? 
\vfill
%Compare to actual data?  1955 = 52, 1960 = 47, 1965 =43, 1970=40, 1975 =37, 1980=34, 1985=31, 1990=28, 1995=25, 2000=23 Really is averaged over the 5-year periods (1950-1955, . . ., 1995-2000)
\end{enumerate}  

% END
\end{enumerate}