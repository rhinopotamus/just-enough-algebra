\documentclass[12pt]{article}
\pagestyle{empty}
\setlength{\parskip}{0in}
\setlength{\textwidth}{6.8in}
\setlength{\topmargin}{-.5in}
\setlength{\textheight}{9.3in}
\setlength{\parindent}{0in}
\setlength{\oddsidemargin}{-.7cm}
\setlength{\evensidemargin}{-.7cm}

\usepackage{amsmath}
\usepackage{amsthm}
\usepackage{amstext}

\usepackage{graphicx}

\begin{document}

\emph{Try taking these practice exams under testing conditions:  no book, no notes, no classmate's help, no electronics (computer, cell phone, television). Give yourself one hour to work and wait until you have tried your best on all of the problems before checking any answers.}
\bigskip

\subsection*{Practice exam 2-- version a}
\bigskip
 \emph{Relax.  You have done problems like these before.  Even if these problems look a bit different, just do what you can.  If you're not sure of something, please ask! You may use your calculator.  Please show all of your work and write down as many steps as you can.  Don't spend too much time on any one problem.  Do well.  And remember, ask me if you're not sure about something.}
 
\bigskip
 
\emph{A few formulas from our book:}

\begin{center}

FORMULAS PRINTED ON EXAM GO HERE

\end{center}

\hspace{-.25in} \hrulefill

\begin{enumerate}
 \item (2.1) The Torkelinsons want to dig a new well for water for their lake cabin.  The company charges \$900 to bring the equipment on site and draw the permit and then \$2 per foot to dig.
\begin{enumerate}
\item What would a 100 foot deep well cost?
\item Name the variables and write an equation relating them.
\item Identify the slope and intercept, along with their units, and explain what each means in terms of the story. 
\item Make a table showing the total cost for a well 100, 250, or 400 feet deep.
\end{enumerate} % MIGHT ALSO BE ON CHAPTER 3 PRACTICE EXAM}

\item (2.1) An insurance \textbf{deductible} is the amount you pay for any claim before the insurance company starts paying.  Lee's automobile insurance starts at \$500, but they take off \$10 for each month where he has no accidents or tickets.
\begin{enumerate}
\item Name the variables and make a table showing the deductible after 6 months, 1 year, or 3 years without an accident or ticket.
\item What is the slope and what does it mean in the story?
\item What is the intercept and what does it mean in the story?
\item Write a linear equation showing how the deductible decreases.
\item Graph the function.
\item When does Lee's deductible actually \textbf{vanish}? (Meaning, when is it \$0.) Estimate the answer from your graph.  Then figure it out.
\end{enumerate} 

\item (2.2) United States ethanol production has been growing exponentially. In 1990, there were .9 billion gallons of ethanol produced.  At that time it was estimated that production would increase 5.5\% per year.
\hfill \begin{footnotesize} Source:  Renewable Fuels Association \end{footnotesize} 
%Link to data: http://www.ethanolrfa.org/industry/statistics/
\begin{enumerate}
\item Name the variables including units. 
\item What is the annual growth factor?
\item Write an equation that describes the dependence.
\item In 2008 actual production of ethanol was 9.0 billion gallons.  Is that production level higher or lower than predicted from your equation?  Explain.
\item What does your equation predict for 2015?  Use successive approximation
\end{enumerate}  % \hfill \emph{Story also appears in 3.4 Exercises}

\item (2.2) Xander grows tomatoes in his garden.  He's noticed that a typical plant yields 5 pounds of tomatoes.  He's been experimenting with the impact of liquid food on plant yield and estimates that each drop increases yield by 14\%.
\begin{enumerate}
\item Calculate the growth factor and write an equation showing how yield for each tomato plant depends on the number of drops of liquid food.  Use $Y$ for the yield (in pounds) and $D$ for the amount of liquid food (in drops).
\item Make a table showing Xander's projections for yield for each tomato plant  if he uses 0, 1, 2, 5, or 10 drops of liquid food.
\item Xander eats about half of the tomatoes as they grow.  The remaining half he uses to make salsa.  Assuming he uses 10 drops of food on a plant which yields the projected amount of tomatoes, and that a pound of tomatoes makes around a pint of salsa, how much salsa will Xander have (from that one plant)?
\item Convert your answer into gallons.  Use $1 \text{ gallon} = 4 \text{ quarts}$ and $1 \text{ quart} = 2 \text{ pints}.$  Ol\'e!
\end{enumerate}

\item (2.3)  Our investment club has been tracking the performance of a biofuel company's stock over the past year.  Using an econometrics software package, we found the equation $$V =.00004W^3 + .01W^2 -.9W + 31$$ 
which describes the value of each share of stock \$$V$ as a function of the week $W$, starting exactly one year ago.  
\begin{enumerate}
\item Complete the following table of values (and check the values already entered):
\begin{center}
\begin{tabular} {|c|c|c |c|c|c |} \hline
$W$ & 0 & 13 & 26 & 39 & 52  \\ \hline
$V$ & 31.00 & 21.08 &\hspace{.4 in}~ & \hspace{.4 in}~ & 16.86 \\ \hline
\end{tabular}
\end{center}
\item According to the table, what is the value of the stock when we began tracking it?  What is it worth now? 
\item We're thinking of buying some stock now, and selling it in 10 weeks.  Does the equation say that's a good idea?  Figure out how much money we earn (or lose) per share on the dea.   \emph{Hint:  10 weeks from now is not $W=10$ because we started counting weeks one year ago.} 
\item Do you want to buy a stock when it's increasing or decreasing in value?

COMBINE THESE:
\item The stock was worth \$31 a year ago when we started tracking.  Now, one year later, it's worth \$16.86 but is on the rise again.  Use successive approximation to estimate when it will return to \$31 in value.  \emph{Report your answer in time from now.}
\item Looking back over the past year, how low did the value of the stock get?  Use successive approximation to estimate to the nearest dollar.
\end{enumerate} 

\item (2.3)  Skye and her sister Clover started a t-shirt printing company.  To produce a particular t-shirt it costs  \$350 in materials and labor to set up a silkscreen and then \$7.50 for each shirt made to cover materials and printing.  The average cost per t-shirt \$$C$ is a function of $N$, the number of t-shirts printed.  The equation for this function is $$C = \frac{350+7.50N}{N}$$

\begin{enumerate}
\item Evaluate this formula when $N=50$ and explain what the value of $C$ you get means in the story.
\item Explain in terms of the story why this function is decreasing.  Sometimes this phenomena is referred to as \textbf{economy of scale}.
\item Make a table showing the average cost per t-shirt if Skye and Clover make 1, 20, 50, 100, or 300 t-shirts.
\item Graph the function.
\end{enumerate}
Skye designs the shirts and runs the press.   Clover is the brains behind sales.  She would like to price the shirts at \$12.95 each.  The sisters will make a profit of \$$P$ where $P = 5.45N-350$.  

SU FIX PARTS NUMBERING.
\begin{enumerate}
\item [(d)]This is a linear equation.  What is the slope, what are its units, and what does it mean in the story?  What is the starting amount, what are its units, and what does it mean in the story?
\item [(e)] How much profit to Skye and Clover make if they sell 100 t-shirts?  What about 300 t-shirts?

COMBINE THESE:
  \emph{In each case use successive approximation to find the answer to the nearest shirt.}

\item The average cost per t-shirt \$$C$ is given by the equation $$C = \frac{350+7.50N}{N}$$ where $N$ is the number of t-shirts printed.  Approximately how many t-shirts would they need to make to keep the average cost per shirt under \$10?  Under \$12.95?
\item The profit  \$$P$ of selling $N$ t-shirts is given by the equation $$P = 5.45N-350$$  How many t-shirts do they need to sell to break even ($P=0$)? To make \$1,000 profit?
\item Is one of your answers to (a) the same as one of your answers to (b)?  They should be.  Explain why.
\end{enumerate}


 \item (2.5)   \begin{enumerate} 
\item Cicely wants to buy a new car that costs \$19,400.  The dealership offers 5.58\% compounded monthly for a 5 year loan.  What will Cicely's monthly payment be? Use the \textsc{Loan Payment Formula}.  
\item What is the equivalent APR Cicely  is paying?  Use the \textsc{Equivalent APR Formula}.   \emph{Don't forget to report the percentage.}
\item Cicely is working on her monthly budget.  She has only \$230 per month left after those car payments.  If she puts that money into a bank account each month earning 2.91\% interest compounded monthly how much will she have after 5 years when the car is paid off?  Use the \textsc{Future Value Annuity Formula}.   
\item In 2011, Cicely was cleaning out the basement and found some savings bonds with face value \$1,600 that matured in 1972 and have been earning 3\% interest compounded monthly ever since.  What were they worth? Use the \textsc{Compound Interest Formula}.  
\end{enumerate}
 
 \item  (2.5)  \begin{enumerate} 
\item Kotoyo's uncle won \$100,000 on a game show.  If he invests it in a fund that's expected to earn 5.7\% interest compounded monthly, how much will he have after 5 years? Use the \textsc{Compound Interest Formula}. 
\item Kotoyo's grandmother has been contributing \$150 a month into a college fund for Kotoyo for the past 8 years.  The account pays 4\% interest compounded monthly.  How much is in the account now? Use the \textsc{Future Value Annuity Formula}.   
\item Kotoyo owes \$8,742 on her credit card.  They charge her 16\% interest compounded monthly.  What would her monthly payment be if she wants to pay it off in 5 years? Use the \textsc{Loan Payment Formula}.  
\item What is the equivalent annual percentage rate (APR) of Kotoyo's credit card? Use the \textsc{Equivalent APR Formula}.  \emph{Don't forget to report the percentage.}
\end{enumerate}

%%% END

\end{enumerate}

SU -- edit and split later

\end{document}

