%!TEX root =  A_WS.tex

\section{A first look at exponential equations  -- Practice exercises}

\begin{enumerate}  

 \item The comprehensive fee at a local private college is \$\text{37,000}.  The fee is projected to increase 5.8\% per year.
\begin{enumerate}
\item Calculate the annual growth factor. \vfill
 \item What do you expect the comprehensive fee will be in five years? \vfill
\item Name the variables, including units, and write an equation describing the dependence. \vfill
\item Make a table of values showing the comprehensive fee now, in 5 years, 10 years, 20 years, and 50 years (even though that's not realistic).  \vfill
\item Draw a graph illustrating the function.
\begin{center}
\scalebox {.8} {\includegraphics [width = 6in] {GraphPaper.jpg}}
\end{center}
\end{enumerate}

\newpage %%%%%%

\item Bunnies, bunnies, everywhere.  They eat the tops of my tulips in early spring and my lilies all summer long.  Back in 2007 there were an estimated  \text{1,800} rabbits in my neighborhood. Rabbits multiply quickly, 13\% per year by one estimate.  

\hfill \emph{Story also appears in 5.1\#3}
\begin{enumerate}
\item Name the variables, including dependency. \vfill
\item Calculate the annual growth factor. \vfill
\item What does this story suggest the rabbit population was in 2010?  In 2013? \vfill
\item Write an equation relating the variables. \vfill
\end{enumerate}  

\newpage %%%%%%

\item A flu virus has been spreading through the college dormitories. Initially 8 students were diagnosed with the flu, but that number has been growing  16\% per day.   

\hfill \emph{Story also appears in  5.1 \#2 and 5.5 textbook}
\begin{enumerate}
\item Calculate the daily growth factor and use it to write an equation describing the spread of the virus.  Don't forget to name the variables too.  \vfill  \vfill
\item Make a table and graph for the six weeks following the initial diagnosis.  (That means use 0, 7, 14, 21, 28, 35, and 42 days.)  \vfill  \vfill
\begin{center}
\scalebox {.8} {\includegraphics [width = 6in] {GraphPaper.jpg}}
\end{center}
\bigskip
\item What is a realistic domain?  That means, for how many days do you think this model is reasonable? To keep a sense of scale, there are \text{1,094} students currently living in the dorms. \vfill
\end{enumerate}  

\newpage %%%%%%

\item My savings account earns a modest amount of interest, the equivalent of .75\% annually.  I have \$\text{12,392.18} in the account now.  
\begin{enumerate}
\item How much interest will I earn this year? \vfill
\item What will my balance be in three years, assuming I neither deposit nor withdraw money? \vfill
\item Name the variables and write an equation relating them. \vfill
\item What would the equation be if I moved all of my money into a certificate of deposit earning the equivalent of .92\%? \vfill
\item What would the equation be if I moved \$\text{10,000} into that certificate of deposit, and kept the rest in savings? 
\emph{Hint:  to find the total balance, add the amounts.} \vfill
\end{enumerate}

\end{enumerate} 

\newpage


\noindent \textbf{When you're done \ldots}

\begin{itemize}
\item [$\Box$] Check your solutions.  Still confused?  Work with a classmate, instructor, or tutor.
\item [$\Box$] Try the \textbf{Do you know} questions.  Not sure?  Read the textbook and try again.
\item [$\Box$] Make a list of key ideas and process to remember under \textbf{Don't forget!}
\item [$\Box$] Do the textbook exercises and check your answers. Not sure if you are close enough? Compare answers with a classmate or ask your instructor or tutor.  
\item [$\Box$] Getting the wrong answers or stuck?  Re-read the section and try again.   If you are still stuck, work with a classmate or go to your instructor's office hours or tutor hours.
\item [$\Box$] It is normal to find some parts of exercises difficult, but if most of them are a struggle, meet with your instructor or advisor about possible strategies or support services.
\end{itemize}





\bigskip

\noindent \textbf{Do you know \ldots} %First_look_exponential

\begin{enumerate} [(a)]
\item How to find the growth factor if you know the percent increase?    
\item How to calculate percent increase in one step?   
\item What makes a function exponential?   
\item The template for an exponential equation? \emph{Ask your instructor if you need to remember the template or if it will be provided during the exam.} 
\item Where the starting value and growth factor appear in the template for an exponential equation?   
\item What the graph of an exponential function looks like? 
\end{enumerate}

\bigskip

\noindent \textbf{Don't forget!}

