%!TEX root =  A_WS.tex

\section{Modeling with linear equations  -- Practice exercises}


\begin{enumerate}
\item A solar heating system costs approximately \$\text{30,000} to install and \$150 per year to run.  By comparison, a gas heating system costs approximately \$\text{12,000} to install and \$700 per year to run.  \hfill \emph{Story also appears in 4.2 Exercises}

\hfill \begin{footnotesize}  Source:  ``Using Algebra'' by Ethan Bolker \end{footnotesize}
\begin{enumerate}
\item What is the total cost for installing and running a gas heating system for 30 years? \vfill  
\item Write a linear equation showing how the total cost for a gas heating system depends on the number of years you run it. \vfill  
\item Write a linear equation showing how the total cost for a solar heating system depends on the number of years you run it. \vfill  
\item If you install and run a solar heating system, how many years can you use it before it costs the same as installing and running a gas heating system for 30 years (your answer to part (a))?  Set up and solve an equation. \vfill   \vfill  
\end{enumerate}

\newpage %%%%%%

\item Since a very popular e-book reader was released, the price has been decreasing at a constant rate.  A blogger developed the following equation representing the price $E$ of the e-book reader in the months $M$ since it was released. $$E = 359 - 12M $$
\begin{enumerate}
\item Make a table of values for the e-book reader price initially, after 10 months, and after 25 months. \vfill  
\item What does the 359 mean in the story and what are its units? \vfill  
\item What does the 12 mean in the story and what are its units? \vfill  
\item Draw a graph illustrating the dependence.  
\begin{center}
\scalebox {.8} {\includegraphics [width = 6in] {GraphPaper.jpg}}
\end{center}

\newpage %%%%%%
~\hspace{-.5in} \emph{The problem continues \ldots}

\item After approximately how many months was the price of the e-book reader expected to be down to  \$200? Set up and solve an equation. \vfill  
\item Sareth decided to purchase a e-book reader when the price fell below \$100.  How many months after its release did the price of the e-book reader fall below that level?  Set up and solve an inequality. \vfill  
\item If you can believe what you read in blogs, the manufacturer will soon be giving away the e-book reader for free, since they make money on the e-book sales themselves.  How many months after it was released would that happen, according to our equation? Set up and solve an equation. \vfill  
\end{enumerate}

\newpage %%%%%%

\item  Can you tell from the table which of these functions are linear?  Use the rate of change to help you decide.  Remember that these numbers may have been rounded.
\begin{enumerate}
\item  Savings bonds from grandpa.  \hfill \emph{Story also appears in 1.2 \#1 and 5.3 \#1} % Yes, that's a clue!

\bigskip
\begin{tabular} {|l|| c| c| c| c| c| c|} \hline
Year & 1962 & 1970 & 1980 & 1990 & 2000 & 2010\\ \hline
Value bond (\$) & 200.00 & 318.77 & 570.87 & \text{1,022.34} & \text{1,830.85} & \text{3,278.77} \\ \hline
\end{tabular}
\vfill


\item Wind chill at 10$^\circ$F.  \hfill \emph{Story also appears in 1.2  \#2}

\bigskip
\begin{tabular} {|l||c|c| c|c|c| c|c|c| c|c|} \hline
Wind (mph)  & 0  & 10  & 20  & 30  & 40  \\ \hline
Wind chill ($^\circ$F) & 10  & -4 & -9 & -12  & -15  \\ \hline
\end{tabular}
\vfill

\item Pizza.  \hfill \emph{Story also appears in 2.4 \#1 and 3.3 \#1}

\bigskip
\begin{tabular} {|l|| c  |c |c|}\hline
Size (inches) & 8  & 14 & 16 \\ \hline
People & 1 & 3 & 4 \\ \hline
\end{tabular} 
\vfill

\item Water in the reservoir. \hfill \emph{Story also appears in 2.1 \#2 and 3.2 Exercises}

\bigskip
\begin{tabular} {|l|| c  |c  |c |c|}\hline
Week & 1 & 5 & 10 & 20 \\ \hline
Depth (feet) & 45.5 & 39.5 & 32 & 17 \\ \hline
\end{tabular}
\vfill


\end{enumerate}

\newpage %%%%%%

\item Plumbers are really expensive, so I have been comparing prices.  James charges \$50 to show up plus \$120 per hour. Jo is just getting started in the business.  She charges \$45 to show up plus \$55 per hour.  Mario advertises ``no trip charge'' but his hourly rate is \$90 per hour. Not to be outdone, Luigi offers to unclog any drain for \$150, no matter how long it takes.  For each plumber, the table lists the corresponding equation and several points.   In each equation,  the plumber charges \$$P$ for $T$ hours of work.   \hfill \emph{Story also appears in 2.1 Exercises}
\begin{center}
\begin{tabular} {|c|| c|| c|| c|| c| } \hline
Plumber & James & Jo &~\quad Mario \quad ~& ~\quad Luigi \quad ~\\  \hline
Equation & $P=50+120T$ & $P=45+55T$ & $P=90T$ & $P=150$ \\ \hline \hline
0 hours & \$50 & \$45 & \$0 & \$150 \\ \hline
2 hours &  \$290 & \$155 & \$180 & \$150 \\ \hline
4 hours & \$530 & \$265 & \$360 & \$150  \\ \hline
\end{tabular}
\end{center}  

\begin{enumerate}
\item Use the points given to plot each of the four lines on the same set of axes.  Label each line with the plumber's name.
\begin{center}
\scalebox {.8} {\includegraphics [width = 6in] {GraphPaper.jpg}}
\end{center}
\item What do you notice about Luigi's line? \vfill
\item List the plumbers in order from steepest to least steep line.  What does that mean in terms of the story? \vfill
\item Now list the plumbers in order from smallest to largest intercept of their line.  What does that mean in terms of the story?  \vfill
\end{enumerate}

\end{enumerate} 

\newpage


\noindent \textbf{When you're done \ldots}

\begin{itemize}
\item [$\Box$] Check your solutions.  Still confused?  Work with a classmate, instructor, or tutor.
\item [$\Box$] Try the \textbf{Do you know} questions.  Not sure?  Read the textbook and try again.
\item [$\Box$] Make a list of key ideas and process to remember under \textbf{Don't forget!}
\item [$\Box$] Do the textbook exercises and check your answers. Not sure if you are close enough? Compare answers with a classmate or ask your instructor or tutor.  
\item [$\Box$] Getting the wrong answers or stuck?  Re-read the section and try again.   If you are still stuck, work with a classmate or go to your instructor's office hours or tutor hours.
\item [$\Box$] It is normal to find some parts of exercises difficult, but if most of them are a struggle, meet with your instructor or advisor about possible strategies or support services.
\end{itemize}





\bigskip

\noindent \textbf{Do you know \ldots} % Modeling linear equations

\begin{enumerate} [(a)]
\item What makes a function linear? 
\item What the slope of a linear function means in the story and what it tells us about the graph? 
\item What the intercept of a linear function means in the story and what it tells us about the graph? 
\item The template for a linear equation?  \emph{Ask your instructor if you need to remember the template or if it will be provided during the exam.}
\item How to write a linear equation given the starting amount (intercept) and the rate of change (slope)?   
\item Where the slope and intercept appear in the template of a linear equation? 
\item What the graph of a linear function looks like? 
\item How to solve a linear equation?  
\item Why the rate of change of a linear function is constant?  
\end{enumerate}

\bigskip

\noindent \textbf{Don't forget!}

