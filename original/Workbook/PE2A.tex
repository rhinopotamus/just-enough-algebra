%!TEX root =  A_WS.tex

\section*{Practice Exam 2A}  
\markright{Practice Exam 2A}
\addcontentsline{toc}{section}{Practice Exam 2A}

Relax.  You have done problems like these before.  Even if these problems look a bit different, just do what you can.  If you're not sure of something, please ask! You may use your calculator.  Please show all of your work and write down as many steps as you can.  Don't spend too much time on any one problem.  Do well.  And remember, ask me if you're not sure about something. \bigskip

\noindent \emph{As you work, make a ``don't forget'' list of any information you need to look up or ask about.} 

\noindent \hrulefill
\bigskip

\begin{enumerate} 

\item United States ethanol production has been growing exponentially. In 1990, there were 0.9 billion gallons of ethanol produced.  At that time it was estimated that production would increase 5.5\% per year.
\hfill \begin{footnotesize} Source:  Renewable Fuels Association \end{footnotesize} 
%Link to data: http://www.ethanolrfa.org/industry/statistics/
\begin{enumerate}
\item Name the variables, including units. \vfill 
\item What is the annual growth factor?  \vfill 
\item Write an equation that describes the function.  \vfill 
\item In 2008 actual production of ethanol was 9.0 billion gallons.  Is that production level higher or lower than predicted from your equation?  Explain.  \vfill 
\item When does your equation predict that ethanol production was (or will be) 9.0 billion gallons? Use successive approximation.  Display your guesses in a table.  Report the actual year.  \vfill   \vfill 
\end{enumerate}  

\newpage %% 

 \item An insurance \textbf{deductible} is the amount you pay for any claim before the insurance company starts paying. Lee's automobile insurance deductible started at \$500, but they take off \$10 for each month where he has no accidents or tickets.  For example, after 1 month his deductible was \$490, after 2 months it was \$480, and so on.
\begin{enumerate}
\item Name the variables.   \vfill  
\item Make a table showing the deductible after 6 months, 1 year, or 3 years without an accident or ticket. \vfill 
\item When would the deductible  \textbf{vanish}? (Meaning, when is it \$0?)  \vfill 
\item Write an equation showing how the deductible decreases.  \vfill 
\item What is the slope and what does it mean in the story?  \vfill 
\item What is the intercept and what does it mean in the story?  \vfill 
\end{enumerate} 

\newpage %% 

\item Our investment club has been tracking the performance of a biofuel company's stock over the past year.  Using an econometrics software package, we found the equation $$V =.00004W^3 + .01W^2 -.9W + 31$$ 
which describes the value of each share of stock \$$V$ as a function of the week $W$, starting exactly one year ago.  
\begin{enumerate}
\item Complete the following table of values. \bigskip % 4 pts
\begin{center}
\begin{tabular} {|c|c|c |c|c|c |} \hline
$W$ & 0 & 13 & 26 & 39 & 52  \\ \hline
$V$ & 31.00 & 21.08 &\hspace{.4 in}~ & \hspace{.4 in}~ & 16.86 \\ \hline
\end{tabular}
\end{center}
\bigskip
\item Draw a graph showing how the value changed during the past year.
\bigskip
\begin{center}
\scalebox {.9} {\includegraphics [width = 6in] {GraphPaper.jpg}}
\end{center}
\bigskip 

\newpage %%
\hspace{-.5in}  \emph{The problem continues \ldots}

\item According to the table, what was the value of the stock when we began tracking it?  What is it worth now? \vfill 
\item We are thinking about buying some stock now, and selling it in 10 weeks.  Does the equation say that's a good idea?  Explain.   \emph{Hint:  10 weeks from now is not $W=10$ because we started counting weeks one year ago.} \vfill \vfill
\item Looking back over the past year, how low did the value of the stock get?  Use successive approximation to estimate to the nearest cent. \vfill \vfill
\end{enumerate} 

\newpage %%

 \item \begin{enumerate} 
\item Cicely wants to buy a new car that costs \$\text{19,400}.  The dealership offers 6.18\% compounded monthly for a 5 year loan.  What will Cicely's monthly payment be? Use the \textsc{Loan Payment Formula}.  \vfill % 6 pts
\item What is the equivalent APR Cicely  is paying?  Use the \textsc{Equivalent APR Formula}.   \emph{Don't forget to report the percentage.} \vfill % 6 pts
\item Cicely is working on her monthly budget.  She has only \$230 per month left after those car payments.  If she puts that money into a bank account each month earning 2.91\% interest compounded monthly how much will she have after 5 years when the car is paid off?  Use the \textsc{Future Value Annuity Formula}.  \vfill  % 6 pts
\item In 2011, Cicely was cleaning out the basement and found some savings bonds with face value \$\text{1,600} that matured in 1972 and have been earning 3\% interest compounded monthly ever since.  What were they worth? Use the \textsc{Compound Interest Formula}. \vfill  % 6 pts
\end{enumerate}

%%%% END

\end{enumerate}

