
\section{Prelude: Percentages} 

\subsection*{Practice exercises}

On each problem, write down what you enter into your calculator and don't forget to write the units on your final answer. You are welcome to calculate the answer step-by-step but also challenge yourself to figure out the answer all at once, not hitting = on your calculator until the very end.


%For exercises look at textbook: 2.2 + 3.4 + Chapter 5
%
%For workbook look at workbook: 2.2 + 3.4 + Chapter 5
%Percents
%Low-fat milk or fat in ground beef 
% RDA of nutrients
%save percentage for goalie in hockey or completion percentage quarterback football
%First gen students or voters for candidate
%
%Percent increase
%Tip on purchase -- have them check Thang's tip
%Interest rate n loan
%Inflation price of food or rent or rent control caps
%Population
%
%Percent decrease
%Ice cap coverage 

\begin{enumerate}

\item As I write this problem, the population of the world is 8,056,959,718 people (just over 8 billion).  It changes by the second, so let's use the round figure of 8,100,000,000.
\begin{enumerate}
\item I read that the population of Brazil accounts for 2.69\% of the world's population.  According to that report, what is the population of Brazil?  Round your answer to the nearest million.  \vfill
\item If the population of the United States is currently around 334,000,000, what percentage of the world's population is in the United States? \vfill
\end{enumerate}

\item In Minneapolis, apartment rent is expected to increase by 16\% next year. \begin{enumerate}
\item Astra lives in a 1-bedroom apartment where they pay \$825 per month in rent.  If their rent increased by 16\% what would their new rent be? \vfill
\item Lucky for Astra, their building is subject to rent stabilization laws and so their rent cannot increase by more than 3\%. What would their new rent be? \vfill
\end{enumerate}

\newpage

\item The corner by my house is dangerous.  One year there were 14 accidents there.  The neighbors got together and petitioned to have 4-way stop signs installed.  
\begin{enumerate}
\item The city estimated that the installed stop signs would reduce accidents at least 40\%.  If that happens, how many accidents would we expect the next year? \vfill
\item The national average shows that the new signs could reduce accidents up to 62\%.  If that happens instead, how many accidents would we expect the next year? \vfill
\item If there were 6 accidents the next year, is that in the range you figured out? What percent decrease does that correspond to? \vfill 
\end{enumerate}

\item My savings account earns a modest amount of interest, the equivalent of .75\% annually.  I have \$12,392.18 in the account now.  \hspace{.7 in} \emph{Story also appears in 2.2\#4}
\begin{enumerate}
\item How much interest will I earn this year? \vfill
\item How much will my account balance be at the end of the year? \vfill
\end{enumerate}

\end{enumerate} % PAUSE

\newpage


\noindent \textbf{When you're done \ldots}

\begin{itemize}
\item [$\Box$] Check your solutions.  Still confused?  Work with a classmate, instructor, or tutor.
\item [$\Box$] Try the \textbf{Do you know} questions.  Not sure?  Read the textbook and try again.
\item [$\Box$] Make a list of key ideas and process to remember under \textbf{Don't forget!}
\item [$\Box$] Do the textbook exercises and check your answers. Not sure if you are close enough? Compare answers with a classmate or ask your instructor or tutor.  
\item [$\Box$] Getting the wrong answers or stuck?  Re-read the section and try again.   If you are still stuck, work with a classmate or go to your instructor's office hours or tutor hours.
\item [$\Box$] It is normal to find some parts of exercises difficult, but if most of them are a struggle, meet with your instructor or advisor about possible strategies or support services.
\end{itemize}





\bigskip

\noindent \textbf{Do you know \ldots}

\begin{enumerate}[(a)]
\item What the words ``per'' and ``cent'' mean in the word ``percent.'' \vfill
\item How to convert a fraction or decimal to a percent? \vfill
\item How to convert a percent to a decimal? \vfill
\item How to calculate a percentage of a number? \vfill
\item How to calculate the result of a percent increase or a percent decrease? \vfill
%\item How to use the distributive property to do percent increase or percent decrease using a single multiplication? %\vfill
\end{enumerate}

\noindent \textbf{Don't forget!}
\vfill \vfill \vfill




