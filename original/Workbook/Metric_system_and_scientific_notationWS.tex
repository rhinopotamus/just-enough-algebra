%!TEX root =  A_WS.tex

\section{Metric prefixes and scientific notation -- Practice exercises}

%\bigskip
%
%Common metric prefixes:
%
%\begin{center}
%\begin{tabular} {lclclcl} 
%\textbf{giga} &$=$& 1 \text{ billion} &$=$& $\text{1,000,000,000}$&$=$&$10^{9}$\\ 
%\textbf{mega} &$=$&1 \text{ million} &$=$&$ \text{1,000,000}$&$=$&$10^{6}$\\
%\textbf{kilo} &$=$&1 \text{ thousand} &$=$&$ \text{1,000}$&$=$&$10^{3}$\\
%\textbf{centi} &$=$&1 in a hundred &$=$&.01&$=$&$10^{-2}$\\
%\textbf{milli} &$=$&1 in a thousand &$=$&.001&$=$&$10^{-3}$\\
%\textbf{micro} &$=$&1 in a million &$=$&.000001&$=$&$10^{-6}$\\
%\textbf{nano} &$=$&1 in a billion &$=$&.000~000~001 &$=$&$10^{-9}$\\
%\end{tabular}
%\end{center}
%
%\newpage

\begin{enumerate}

\item The GDP (gross domestic product) of the United States was approximately \$\text{15,596} billion in 2011 and the population of the United States was approximately 0.313 billion that year.  \hfill \begin{footnotesize} Source:  U.S.\ Bureau of Economic Analysis, U.S.\ Census Bureau\end{footnotesize}
%http://www.tradingeconomics.com/united-states/gdp
%http://www.census.gov/main/www/popclock.html
\begin{enumerate}
\item Writing the population as 0.313 billion seems strange.  A more natural unit would be millions.  Rewrite the population in millions of people. \vfill
\item Rewrite the population in people, both in normal decimal notation (that means with all the 0s) and in scientific notation. \vfill
\item It also seems strange to write the GDP as \$\text{15,596} billion.   A more natural unit would be \textbf{trillions} where
$$1 \text{ trillion} =  \text{ 1,000,000,000,000}$$
Rewrite the GDP in trillions of dollars. \vfill
\item Rewrite the GDP in dollars, both in normal decimal notation and in scientific notation. \vfill
\item Calculate the GDP \textbf{per capita} (meaning per person) by dividing the GDP in dollars by the population in people.  Express your answer in \$/person. \vfill
\item For practice, repeat your calculation using the numbers in scientific notation.  
 
 \emph{Because $\times$ and $\div$ are at the same level in the order of operations, you need to put parentheses around each number in scientific notation before dividing.} \vfill
\end{enumerate}

\newpage %%%%%%

\item Edgar recently changed the cleaning bag on his vacuum cleaner.  He became curious about how many particles of dust were in the bag.  Edgar did a little research online and found out that the mass of a dust particle is .000~000~000~753 kilograms.  

(The strange-looking spaces are to help you see that there are 9 zeros in the number.)
\begin{enumerate}
\item Write the mass of a dust particle in scientific notation. \vfill
\item Recall that 

\begin{center}
\begin{tabular} {lclclcl} 
\textbf{kilo} &$=$&1 \text{ thousand} &$=$&$ \text{1,000}$&$=$&$10^{3}$\\
\textbf{milli} &$=$&1 in a thousand &$=$&.001&$=$&$10^{-3}$\\
\textbf{micro} &$=$&1 in a million &$=$&.000001&$=$&$10^{-6}$\\
\textbf{nano} &$=$&1 in a billion &$=$&.000~000~001 &$=$&$10^{-9}$\\
\end{tabular}
\end{center}

Express the mass of a dust particle in each of the given units.  

\begin{enumerate}
\item  grams \vfill
\item milligrams (mg) \vfill
\item micrograms ($\mu$g)  \vfill
\item nanograms (ng)  \vfill
\end{enumerate}
%% SU you should add this negative exponent description to the section too. Maybe a list of all the prefixes w/ it???
\item Edgar determined that the full vacuum bag weighed 5 pounds. How many dust particles were in the bag?  (I am already sneezing.) Use $1 \text{ kilogram} \approx 2.2 \text{ pounds}$. Express your answer in scientific notation.   \vfill \vfill
\end{enumerate} 

\newpage %%%%%%

\item The list shows the (approximate) mass of the planets in our solar system.
\begin{center}
\begin{tabular} {ll}
Earth & $5.972 \times 10^{24}$ kg \\
Jupiter &  $1.899 \times 10^{27}$ kg  \\
Mars & $6.417 \times 10^{23}$ kg \\ 
Mercury & $3.302 \times 10^{23}$ kg \\
Neptune & $1.024 \times 10^{26}$ kg \\
Saturn & $5.685 \times 10^{26}$ kg \\
Uranus & $8.681 \times 10^{25}$ kg \\
Venus & $4.868 \times 10^{24}$ kg \\ 
\end{tabular}
\end{center}
\hfill \begin{footnotesize} Source:  Wikipedia (Solar System) \end{footnotesize}
% http://en.wikipedia.org/wiki/Solar_System
\begin{enumerate}
\item Write the mass of Earth and the mass of Mars in standard decimal notation.  Which is heavier?\vfill
\item List the planets from heaviest (largest mass) to lightest (smallest mass).\vfill 
\item The mass of astronomical bodies are sometimes measured in \textbf{Jupiter mass} abbreviated $M_J$ where $1 M_J = 1.899 \times 10^{27}$ kg.  Express Earth's mass in $M_J$.

 \emph{Because $\times$ and $\div$ are at the same level in the order of operations, you need to put parentheses around each number in scientific notation before dividing.} \vfill
\end{enumerate}

\newpage %%%%%%

\item Souksavanh is setting up a patient's intravenous (IV) medication. She sets the drip at 42 drops/minute.  The drip chamber size is 20 drops/mL.  Recall

\begin{center}
\begin{tabular} {lclclcl} 
\textbf{milli} &$=$&1 in a thousand &$=$&.001&$=$&$10^{-3}$\\
\textbf{micro} &$=$&1 in a million &$=$&.000001&$=$&$10^{-6}$\\
\end{tabular}
\end{center}
\begin{enumerate}
%\item Souk needs to know a few conversions.
%\begin{enumerate}
%\item How many milliliters (mL) are in a liter? \bigskip
%\item How many microliters ($\mu$L) are in a liter? \bigskip
%\item How many milligrams (mg) are in a gram? \bigskip
%\end{enumerate}
%\hspace{-.35in}  Use these numbers to answer the following questions. %HSPACE
\item At what rate is the IV fluid being delivered to Souk's patient?  Answer in mL/min (millileters per minute). \vfill
\item How fast is the drip measured in $\mu$L/sec (microliters per second)? \vfill
\item If the drip bag holds 1 liter, how long will it take the drip to run?  Express your answer in hours and minutes.\vfill
\item The concentration of medication is 1.7 mg/mL (milligrams per milliliter).  How much medication is in the 1 liter bag?  Convert your answer to grams.  Explain what you notice.\vfill
\item At what rate is the medication being delivered to Souk's patient?  Answer in g/min (grams per minute).\vfill
\end{enumerate}

\end{enumerate} 

\newpage


\noindent \textbf{When you're done \ldots}

\begin{itemize}
\item [$\Box$] Check your solutions.  Still confused?  Work with a classmate, instructor, or tutor.
\item [$\Box$] Try the \textbf{Do you know} questions.  Not sure?  Read the textbook and try again.
\item [$\Box$] Make a list of key ideas and process to remember under \textbf{Don't forget!}
\item [$\Box$] Do the textbook exercises and check your answers. Not sure if you are close enough? Compare answers with a classmate or ask your instructor or tutor.  
\item [$\Box$] Getting the wrong answers or stuck?  Re-read the section and try again.   If you are still stuck, work with a classmate or go to your instructor's office hours or tutor hours.
\item [$\Box$] It is normal to find some parts of exercises difficult, but if most of them are a struggle, meet with your instructor or advisor about possible strategies or support services.
\end{itemize}





%\bigskip

\noindent \textbf{Do you know \ldots} % Metric system and scientific notation

\begin{enumerate} [(a)]
\item How to calculate powers on your calculator?
\item What  million, billion, and trillion mean?  
\item Why metric prefixes are used?  
\item What common metric prefixes (mega, giga, kilo, centi, milli, micro, nano) mean? 

\emph{Ask your instructor which prefixes you need to remember, and whether any prefixes will be provided during the exam.} 
\item Why scientific notation is used?  
\item The standard format for scientific notation?  
\item What kinds of numbers have a positive order of magnitude, and which have a negative order of magnitude?
\item How to convert between decimal notation and scientific notation?  
\item How your calculator reports numbers in scientific notation, and what (might be) different when you're reporting that number?  
%\item How to enter numbers written in scientific notation into your calculator? 
\item The usual order of operations (PEMDAS) and how to use parentheses when you want a different order?
\end{enumerate}

\bigskip

\noindent \textbf{Don't forget!}


