

\section{Growth factors}

Obesity among children ages 6-11 continues to increase.  From 1994 to 2010, the proportion of children classified as obese rose from an average of 1.1 out of every ten children in 1994 to around 2 out of every ten children in 2010.
%citation: http://www.berksmontnews.com/articles/2010/10/20/community_connection/news/doc4cbed761c7fcc149792696.txt  They credit centers for disease control.
% CDC report is http://www.cdc.gov/nchs/data/hestat/obesity_child_07_08/obesity_child_07_08.htm
Assuming that the prevalence of childhood obesity increases exponentially, what is the annual percent increase and what does the equation project for the year 2020? Well, unless we are able to make drastic improvements in how children eat and how much they exercise.

Because we are told obesity is increasing exponentially we can use the template for an exponential equation.  
$$\text{dep }=\text{ start } \ast \text{growth factor}^{\text{indep}}$$
The variables are 
\begin{center}
\begin{tabular} {l} 
$C=$ obese children (out of every ten) $\sim$ dep \\ 
$Y =$ year (years since 1994) $\sim$ indep \\
\end{tabular}
\end{center}
The starting amount  is 1.1 children out of every ten in 1994 so our equation is of the form $$C = 1.1 \ast g^Y$$
Trouble is we don't actually know what the growth factor $g$ is.  Yet.  

We do know that in 2010 we have $Y = 2010 - 1994 = 16$ years and $C = 2$.  We can put those values into our equation to get $$1.1 \ast g^{16}=2$$
No good reason for switching sides, just wanted to have the variable on the left.
That's supposed to be true but we don't know what number $g$ is so we can't check.  Argh.  

Oh, wait a minute.  The only unknown in that equation is the growth factor $g$.  What if we solve for $g$?  First, divide each side by  1.1 to get
$$ \frac{\cancel{1.1}\ast g^{16}}{\cancel{1.1}} = \frac{2}{1.1} $$
which simplifies to $$g^{16} = \frac{2}{1.1} = 2 \div 1.1= 1.818181818\ldots$$
Since we want to solve for the base (not the exponent), we have a power equation.  We use the \textsc{Root Formula} with power $n=16$ and value $v=1.818181818$ to get
$$g = \sqrt[n]{v} = \sqrt[16]{1.818181818} = 16 \sqrt[x]{\text{\raisebox{.6em}{~\text{  }}}} 1.818181818 = 1.038071653 \approx 1.0381$$  

Want a quicker way to find the growth factor?  Forget the entire calculation we just did.  %From ``Yet''
It all boils down to two steps: $$\frac{2}{1.1} = 2 \div 1.1 = 1.818181818\ldots$$
and then $$g =  \sqrt[16]{1.818181818} = 16 \sqrt[x]{\text{\raisebox{.6em}{~\text{  }}}} 1.818181818 = 1.038071653 \approx 1.0381$$
We can do this calculation all at once as
 $$g =  \sqrt[16]{\frac{2}{1.1}} = 16 \sqrt[x]{\text{\raisebox{.6em}{~\text{  }}}} (2 \div 1.1)= 1.038071653 \approx 1.0381$$
Notice we added parentheses because the normal order of operations would do the root first and division second.  We wanted the division calculated before the root.  

Here's the easy version in a formula.

 \bigskip
 \framebox{
 \begin{minipage}[c]{.85\textwidth}  
~ \bigskip \\  \textsc{Growth Factor Formula} \\ ~\\
If a quantity is growing (or decaying) exponentially, then the growth (or decay) factor is 
$$\displaystyle g = \sqrt[t]{\frac{a}{s}}$$ where $s$ is the starting amount and $a$ is the amount after $t$ time periods. \\
\end{minipage}
}
  \bigskip

\noindent  Let's check.  In our example, the number of obese children grew from the starting amount of $s=1.1$ to the amount of $a=2$ after $t=16$ years.  By the \textsc{Growth Factor Formula} we have 
$$g =  \sqrt[t]{\frac{a}{s}} \\
=  \sqrt[16]{\frac{2}{1.1}} \\
= 16 \sqrt[x]{\text{\raisebox{.6em}{~\text{  }}}}  (2 \div 1.1) \\
= 1.038071653 \approx 1.0381 \quad \checkmark$$

Either way, we knew from the beginning that our equation was in the form
$C = 1.1 \ast g^Y$.  Now that we found the growth factor $g \approx 1.0381$ we get our final equation $$C = 1.1 \ast 1.0381^Y$$
For example, we can check that in 2010, we have $Y=16$ still and so 
$$C = 1.1 \ast 1.0381^{16} = 1.1 \times 1.0381 \wedge \underline{16} =  2.000874004 \approx 2 \quad \checkmark$$ 

You might wonder why we didn't just round off and use the equation $$C = 1.1 \ast 1.04^Y$$ 
Look what happens when we evaluate at $Y=16$ then.  We would get
$$C = 1.1 \ast1.04^{16} = 1.1 \times 1.04 \wedge \underline{16} = 2.06027937 \approx 2.1$$
Not a big difference (2.1 vs.\ 2.0) but enough to encourage us to keep extra digits in the growth factor in our equation.  Lesson here is:  don't round off the growth factor too much.

Back to the more reliable equation $$C = 1.1 \ast 1.0381^Y$$
We can now answer the two questions. First, in 2020 we have $Y = 2020-1994 = 26$ and so $$C = 1.1 \ast 1.0381^{26} = 1.1 \times 1.0381 \wedge \underline{26} = 2.908115507 \approx 2.9$$
According to our equation, by 2020 there would be approximately 2.9 obese children for every ten children.

The other question was what the annual percent increase is.  Think back to an earlier example.  Remember that Jocelyn was analyzing health care costs?  They began at \$2.26 million and grew 6.7\% per year.  She had the equation 
$$H=2.26\ast1.067^Y$$
So the growth factor $g=1.067$ in the equation came from the growth rate $r=6.7\%=.067$.
Our equation modeling childhood obesity is $$C = 1.1 \ast 1.0381^Y$$ 
The growth factor of $g=1.0381$ in our equation must come must come from the growth rate $r= .0381=3.81\%$. Think of it as converting to percent $1.0381 = 103.81 \%$ and then ignoring the 100\% to see the 3.81\% increase.
Childhood obesity has increased around 3.81\% each year.  Well, on average.
 
Here's the general formula relating the growth rate and growth factor.

 \bigskip
 \framebox{
 \begin{minipage}[c]{.85\textwidth}  
~ \bigskip \\  \textsc{Percent Change Formula:} \hfill (updated version)
\begin{itemize}
\item   If a quantity changes by a percentage corresponding to growth rate $r$, then the growth factor is $$\displaystyle g=1+r$$
\item If the growth factor is $g$, then the growth rate is $$r = g-1$$ ~
\end{itemize}
\end{minipage}
}
\bigskip

\noindent Let's check.  We have $g=1.0381$ and so the growth rate is 
$$r=g-1 = 1.0381-1 = .0381= 3.81\%$$
Not sure we really need these formulas, but there you have it.

By the way, formula works just fine if a quantity decreases by a fixed percent. One example we saw was Joe, who drank too much coffee.  The growth (or should I say decay) factor was $g=.87$.  That corresponds to a growth (decay) rate of 
$$r=g-1=.87-1=-.13=-13\%$$
Again, the negative means that we have a percent decrease.

%Is that too many formulas all at once?  For the purpose of this section, you can forget about the \textsc{Root Formula} and just use \textsc{Growth Factor Formula} instead.  

%Both the \textsc{Growth Factor Formula} and the \textsc{Percent Change Formula} tell us the growth (or decay) factor but they apply in separate situations.  We use the \textsc{Growth Factor Formula} when we know the starting and ending amount and are told the equation is exponential.  It's a good question how we know it's exponential in those situations, but don't forget there's often a combination of science and data behind the scenes.  We use the (much easier) \textsc{Percent Change Formula} when we are told (or are looking for) the percent increase or decrease.  

%Don't forget that sometimes the story tells us the growth or decay factor in the story.  Like where three of you make a pact to each invite 10 friends to join your online group, and they each invite 10 friends, and so on.  There the equation was $$F = 3\ast10^N$$ which means the growth factor was $g=10$.  An example where the story told us the decay factor was with sheets of glass filtering out the light, only letting 75\% through.  There the equation was $$L = 100\ast .75^S$$ which means the decay factor was $g=.75$.  %SU check that these stories are actually in teh WORKSHEETS!

%\newpage

%%\section{Growth factors}

 \begin{center}
\line(1,0){300} %\line(1,0){250}
\end{center}

\section*{Homework}

\noindent \textbf{Start by doing Practice exercises \#1-4 in the workbook.}

\bigskip

\noindent \textbf{Do you know \ldots}

\begin{itemize}
\item How to find the growth/decay factor given the starting amount and another point of information? 
\item How to find the growth/decay factor given the doubling time or half-life? 
\item When we use the \textsc{Percent Change Formula}, and when we use the \textsc{Growth Factor Formula} instead?  \emph{Ask your instructor if you need to remember the \textsc{Percent Change Formula} and \textsc{Growth Factor Formula} or if they will be provided during the exam.}
\item How to evaluate the \textsc{Percent Change Formula} and \textsc{Growth Factor Formula} using your calcuator? 
\item How to read the starting amount and percent increase/decrease from the equation? 
 \item[~] \textbf{If you're not sure, work the rest of exercises and then return to these questions.  Or, ask your instructor or a classmate for help.} 
\end{itemize}

\subsection*{Exercises}

\begin{enumerate} 
\setcounter{enumi}{4}

\item Estimates for childhood obesity for 2010 were revised to 2.1 out of every ten children.  (The 1994 figure of 1.1 out of every ten children remains accurate.)
\begin{enumerate}
\item Calculate the revised growth factor.  What is the revised percent increase?
\item Revise your equation.
\item Use your new equation to project childhood obesity rates for 2020.
\item Graph both the original and revised estimates on the same set of axes.  
\end{enumerate}

\item For each equation, find the growth rate (percent increase or percent decrease) and state the units. (For example, something might ``grow 2\% per year'' while something else might ``drop 7\% per hour'')  
\begin{enumerate} 
\item The light $L\%$ that passes through panes of glass $W$ inches thick is given by the equation
$$L = 100\ast 0.75^W$$
\hfill \emph{Story also appears in 2.4 and 3.4 Exercises}

\item The population of bacteria ($B$) in a culture dish after $D$ days is given by the equation $$B=2,000\ast 3^D$$
\hfill \emph{Story also appears in 5.2 Exercises}

\item The remaining contaminants ($C$ grams) in a waste water sample after $M$ months of treatment is given by $$C=8 \ast 0.25^M$$
\hfill \emph{Story also appears in 5.2 Exercises}
\end{enumerate}

\item Years ago, Whitney bought an antique mahogany table worth \$560.  Now, 30 years later, she had the table appraised for \$3,700.  
\begin{enumerate}
\item Calculate the annual growth factor, assuming the value of Whitney's table has increased exponentially.
\item What should she expect the set to be worth in another 10 years? As part of your work, name the variables and write an equation relating them.
\end{enumerate}

\item The opiate drug morphine leaves the body quickly.  After 72 hours about 10\% remains.  A patient receives 100 mg of morphine.
\begin{enumerate}
\item How much morphine will remain in the patient's body after 72 hours?
\item Convert 72 hours to days.
\item Find the daily decay factor using the \textsc{Growth Factor Formula}.
\item What is the corresponding percent decrease?
\item Name the variables and write an equation relating them.  Check that 72 hours gives you the same answer as in part (a).
\item What is the half-life of morphine?  Set up and solve an appropriate equation.
\item Draw a graph showing this patient's morphine levels for 10 days following the injection.
\end{enumerate}

\item Unemployment figures were just released.  At last report there were 20,517 unemployed adults and now, 10 months later, we have 39,061 unemployed adults.  \begin{enumerate}
\item Calculate the monthly growth factor, assuming unemployment increases exponentially.
\item Write an equation relating the variables.
\item According to your equation, what is the expected number of unemployed adults 6 months from now.  \emph{Notice:  the report was issued 10 months ago.}
\item Make a table of values and draw a graph showing the number of unemployed adults for the past 10 months and the next 2 years.
\end{enumerate}

\item Wetlands help support fish populations, various plant and animal populations, control floods and erosion from nearby lakes and streams, filter water, and help preserve our supply of ground water. 
 Minnesota wetlands acreage in 1850 was 18.6 million acres.  By 2003, that number had dropped to 9.3 million acres. 
 
 \hfill \begin{footnotesize} Source:  Minnesota Department of Natural Resources \end{footnotesize}

 \begin{enumerate}
\item Assuming the acreage decreased exponentially, name the variables, find the annual decay factor and  write an exponential equation showing how Minnesota wetlands have decreased.
\item With some effective management, many wetlands have been restored.  By 2012, it's up to about 10.6 million acres.  Assuming acreage has increased exponentially from 2003, name the variables (you may now want to start the years in 2003), find the growth factor and write an exponential equation showing how Minnesota wetlands have been restored.   
\end{enumerate}

\end{enumerate}

