%!TEX root =  A_WS.tex

\section*{Practice Exam 1A}  
\markright{Practice Exam 1A}
\addcontentsline{toc}{section}{Practice Exam 1A}

Relax.  You have done problems like these before.  Even if these problems look a bit different, just do what you can.  If you're not sure of something, please ask! You may use your calculator.  Please show all of your work and write down as many steps as you can.  Don't spend too much time on any one problem.  Do well.  And remember, ask me if you're not sure about something. \bigskip

\noindent \emph{As you work, make a ``don't forget'' list of any information you need to look up or ask about.} 

\noindent \hrulefill
\bigskip

\begin{enumerate}
 
 \item Arva and Ellie began hiking at an elevation of 1,500 feet and climbed at the steady rate of 600 vertical feet per hour. 
\begin{enumerate}
\item Make a table showing their elevation after 1 hour, 2 hours, and 5 hours. \vfill \vfill
\item Name the variables, including units. \vfill \vfill
\item Explain the dependence using a sentence of the form ``\underline{~\quad} is a function of \underline{~\quad}.'' \vfill
\item Is the function increasing or decreasing? \vfill
\item How long does it take them to reach 5,300 feet up?  Try to figure out the answer in hours and minutes (H:MM format). \vfill \vfill \vfill
\end{enumerate}

\newpage

\item The table shows Henry's weight as a baby.
\begin{center}
\begin{tabular} {|l||c|c|c|} \hline
Age (weeks) & 0 & 12 & 15 \\ \hline
Weight (pounds) & 8 & 14 & 16 \\ \hline
\end{tabular}
\end{center}
\begin{enumerate}
\item How much weight did Henry gain, on average, each week during his first 12 weeks? \vfill
\item During which time interval was Henry gaining weight faster?  \emph{Explain.} \vfill
 \item Identify the variables, including units and dependence. \vfill
 \item Draw a graph illustrating the dependence.  Choose a scale that shows up to 20 weeks and 20 pounds. \bigskip
\begin{center}
\scalebox {.8} {\includegraphics [width = 6in] {GraphPaper.jpg}}
\end{center}

\bigskip
\item What might you guess for Henry's weight at 20 weeks?   \vfill
\end{enumerate} 

\newpage

\item Pramesh's new car used 20.5 gallons of gas for a 715 mile trip. 
\begin{enumerate}
\item How many miles per gallon (mpg) does his car get? \vfill
\item At that rate, how many gallons of gas would Pramesh use on his 3,200 mile cross-country trip?  \vfill
\item If gas costs \$3.799/gallon, how much will gas for that trip cost?  \vfill
\end{enumerate}

\newpage

\item Ndwiga is reading an article in the paper about atoms.  From his physics textbook he discovered that the size of an atom is .142 nanometers.  (That's 0.142 nanometers.)
\begin{enumerate}
\item Write the size of an atom in meters.  Use $1 \text{ meter} = \text{1,000,000,000 nanometers}$.   Write your answer in usual decimal notation and in scientific notation. \vfill \vfill
\item Ndwiga would like to know how many atoms across this sheet of paper which is 8.5 inches wide. Use that $1 \text{ inch} \approx 2.54 \text{ cm}$ and $1 \text{ meter} = 100 \text{ cm}$.  Express your final answer in billions of atoms. \vfill \vfill \vfill
\end{enumerate}

%%%% END

\end{enumerate}

