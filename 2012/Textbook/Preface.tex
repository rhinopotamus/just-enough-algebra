\section*{Preface}
\addcontentsline{toc}{section}{Preface}

In 1994, my colleagues at Augsburg College (now Augsburg University) and I had a vision for a new course to replace our intermediate algebra course.  We wanted a college level course that would serve primarily as preparation for quantitative courses across the curriculum. The framing question that led to our curricular adventure of the past nearly two decades was 
\begin{quote}  What algebra do college students need to know, and how can we make it relevant to their future studies, their lives as citizens, and their everyday life?
\end{quote}
From these questions \emph{Just enough algebra} was born.  

As you will see, everything we do is in some applied context.  
Our choice to focus primarily on linear and exponential models; to emphasize verbal, numerical, and graphical interpretation of functions; and to include only the most essential symbolic techniques  align well with curricular guides from the MAA\footnote{\emph{Curricular Guide}, Committee on the Undergraduate Mathematics Program (CUPM) and \emph{Curriculum Foundations Project:  Voices of the Partner Disciplines}, Curriculum Renewal Across the First Two Years (CRAFTY), Mathematical Association of America (MAA), 2004 and CRAFTY's \emph{Recommendations for College Algebra}, 2007} 
and AMATYC\footnote{\emph{Crossroads in Mathematics Standards for Introductory College Mathematics Before Calculus}, American Mathematical Association of Two-Year Colleges (AMATYC), 1995 and the follow up ``Beyond Crossroads'' report, AMATYC, 2006.}.
More importantly, it works.  Student learn a lot in this course.  They are ready for what comes next.  And, they enjoy it.


\section*{To the student}
This textbook is written for you.  Read the narrative examples, listen to examples in class, try the practice exercises (in the accompanying workbook), check with classmates or your instructor, then work the exercises in this textbook, then chat with classmates again, then do more problems, \ldots.  Well, you get the idea:  the best way to learn mathematics is to do it yourself.  I hope you enjoy the course.  I know you will learn a lot of useful algebra.  I believe it will change how you see mathematics.

\section*{To the instructor}
This textbook is written for students.  That means you won't find list of student learning objectives anywhere, although I'm sure you can infer them from the student-focused ``Do you know \ldots'' questions  in each section.  While the narrative examples develop the main theme of each section, I've deliberately left some variations for the practice exercises.   
 These practice exercises are designed to be started during class and are printed in a separate workbook for that purpose.   Hand-written solutions to the practice exercises are available (in electronic format) for students to check their work, whether in class or at home.

My greatest success in teaching this course has been to give students room to figure things out for themselves, so try to resist the temptation to show them one of everything.   Listen to your students and help them understand the algebra in their own vocabulary.  You will be impressed.

\section*{Acknowledgments}

Thanks, first, to the thousands of students who have taken this course.  Their creative approaches to learning mathematics; their unedited criticism and challenge; their often surprising enthusiasm for the course; their patience tried by countless typos and outright mistakes; and their perpetually novel insights have humbled me and challenged everything I thought I knew about teaching and learning mathematics.  They have inspired me time and time again.  I am grateful that they have allowed me to make a difference in their lives.

Thanks, next, to my mathematics colleagues at Augsburg: Mathew Foss (now at North Hennepin Community College), who taught from the very first edition of the textbook back in 1997 and who collaborated in writing earlier versions; Matt Haines, Alyssa Hanson, Rich Flint, and the dozens of other professors who have taught the course over the years from various earlier editions of the textbook; John Zobitz and Jody Sorensen, who edited and created more exercises for this 2012 edition;  and student helpers Ashley Gruhlke and Emma Winegar.

Thanks, also, to my colleagues across campus for allowing us to try something completely different and to my mathematics colleagues nationally for spurring me on.  During the first few years I taught from \emph{Using algebra} by Ethan Bolker.  Much of my approach and probably more examples or exercises than I realize are derived from his vision and from the subsequent text by his colleagues Linda Kime and Judy Clark at University of Massachusetts, Boston.  

A special thanks to Dean Barbara Farley, Augsburg's Center for Teaching and Learning, and Augsburg's Undergraduate Research and Graduate Opportunity program for supporting my work through sabbaticals, travel grants, and summer research grants for both me and students.


