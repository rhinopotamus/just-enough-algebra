%!TEX root =  A_WS.tex

\section{Logistic and other growth models -- Practice exercises}  

\begin{enumerate}

\item Corn farmers say that their crop is healthy if it is ``knee high by the Fourth of July.''  An equation that relates the height $H$ (in inches) of the corn crop $D$ is days since May 1 is $$H=106-100 \ast .989^{D} $$ %Saturating

\begin{enumerate}
\item According to this equation, how high is corn projected to be on June 1 (day 31)? \vfill
\item According to this equation, how high is corn projected to be on the Fourth of July (day 64)?  Is that ``knee high'' (18 inches tall)?  \vfill
\item With stronger corn these days, the rule ought to be ``chest high (52 inches) by the Fourth of July."  According to this equation, when is the corn projected to be that tall?  Use successive approximation to answer. \vfill
\item The corn matures in 110 days.  How tall will it be then?  \vfill
\item Draw a graph of the function.  Include when $D=0$.
\begin{center}
\scalebox {.8} {\includegraphics [width = 6in] {GraphPaper.jpg}}
\end{center}
\end{enumerate}

\newpage %%%%%%

\item An alternative equation for corn height is  $$H = \frac{200}{1+70\ast.965^D}$$ %Logistic
\begin{enumerate}
\item According to this new equation, how high is corn projected to be on June 1 (day 31)? \vfill
\item According to this new equation, how high is corn projected to be on the Fourth of July (day 64)?  Is that ``knee high'' (18 inches tall)?  \vfill
\item According to this new equation, on approximately what date is the corn projected to be ``chest high''  (52 inches tall)?  Use successive approximation to answer.  \vfill  \vfill
\item The corn matures in 110 days.  How tall will it be then, according to this new equation?  \vfill 
\item Add the graph of this function to your graph of the original equation on the previous problem.  Again, include when $D=0$.
\end{enumerate}

\newpage %%%%%%

\item Back in 1975 when my aunt and uncle bought their house upstate New York, there was a small pond in the yard.  They enlarged it and stocked it with 10 small fish. The number of fish $F$ increased over time, approximately according to the equation
$$ F=\frac{\text{1,000}}{1+99 \ast .65^Y}$$
where $Y$ measures the years since 1975.

\begin{enumerate}
\item Make a table showing the fish population in 1975, 1990, 2000, and 2013. \vfill
\item By the time there were over 500 fish in the pond, you could catch them with your bare hands.  In approximately what year did that happen? \vfill
\item In approximately what year did the fish population reach its capacity?  Use successive approximations and display your calculations in a table. \vfill

\end{enumerate}

% \begin{enumerate}
%\item How much radiation was detected at the start of the leak? At 8 hours later?  10 hours after?  16 hours?  24 hours? \vfill
%\item Approximately when did the radiation level off? (Display your work in a table.) What was the largest amount of radiation at that time? \vfill
%\item The normal level of radiation that a person is exposed to around 2.4 mSv during an entire  year.  What is that normal level of radiation measured in mSv/day?  Use $1 \text{ year} = 365 \text{ days}$. \hfill \begin{footnotesize} Source:  Wikipedia (Sievert) \end{footnotesize}  \vfill
%\item At its largest amount (where it leveled off), did the radiation the exceed normal daily levels?  If so, by how many times normal?  
%
%\emph{That means divide your answer to (b) by your answer to (c).} \vfill
%\end{enumerate}
% Ask John if it's 1.62 or really .162.  They will be a little freaked out Helpful facts: Chest x-ray .05 milliSieverts, 2.4 milliSieverts = average amount of  natural radiation
%near the Fukushima Daiichi 

\newpage %%%%%%

\item Jason works at a costume shop selling Halloween costumes.  The shop is busiest during the fall before Halloween.  An equation that describes the number of daily visitors $V$ the shop receives $D$ days from August 31 is the following:
$$ V=\frac{430}{1+701\ast .81^D}$$ % Logistic
An alternative equation is $$V = 700 - 690 \ast .985^D$$ %Saturating
\begin{enumerate}
\item Make a table showing what each equation predicts for August 31, September 15, September 30, October 15, October 25, October 28, and October 31. 

\emph{Hint:  those days are numbered 0, 15, 30, 45, 55, 58, and 61.} \vfill \vfill
\item Graph both functions on the same set of axes.
\begin{center}
\scalebox {.8} {\includegraphics [width = 6in] {GraphPaper.jpg}}
\end{center}
\bigskip
\item Which function is more consistent with a major advertising campaign during the second week of September?  Explain. \vfill
\end{enumerate}


\end{enumerate}

\newpage


\noindent \textbf{When you're done \ldots}

\begin{itemize}
\item [$\Box$] Check your solutions.  Still confused?  Work with a classmate, instructor, or tutor.
\item [$\Box$] Try the \textbf{Do you know} questions.  Not sure?  Read the textbook and try again.
\item [$\Box$] Make a list of key ideas and process to remember under \textbf{Don't forget!}
\item [$\Box$] Do the textbook exercises and check your answers. Not sure if you are close enough? Compare answers with a classmate or ask your instructor or tutor.  
\item [$\Box$] Getting the wrong answers or stuck?  Re-read the section and try again.   If you are still stuck, work with a classmate or go to your instructor's office hours or tutor hours.
\item [$\Box$] It is normal to find some parts of exercises difficult, but if most of them are a struggle, meet with your instructor or advisor about possible strategies or support services.
\end{itemize}





\bigskip

\noindent \textbf{Do you know \ldots} % Logistic growth

\begin{enumerate} [(a)]
\item Why we might use a logistic or saturation model, instead of an exponential model?
\item The difference between a logistic and saturation model?

\item What the limiting value of a logistic function means in the story and what it tells us about the graph? 
\item How to find the limiting value of a logistic function?  
\item What the graph of a logistic function looks like? 

\item What the limiting value of a saturation function means in the story and what it tells us about the graph? 
\item How to find the limiting value of a saturation function?  
\item What the graph of a saturation function looks like? 
\end{enumerate}

\bigskip

\noindent \textbf{Don't forget!}


