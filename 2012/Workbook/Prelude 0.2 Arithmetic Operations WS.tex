
\section{Prelude: Arithmetic Operations} 

\subsection*{Practice exercises}

On each problem, write down what you enter into your calculator and don't forget to write the units on your final answer. You are welcome to calculate the answer step-by-step but also challenge yourself to figure out the answer all at once, not hitting = on your calculator until the very end.

\begin{enumerate}

\item Tensia loves to garden but can't quite keep up with how many cucumbers are growing.  
\begin{itemize}
\item At the start of the week she had 8 cucumbers in her refrigerator. 
\item Her son, N\'estor took 3 home with him after dinner on Monday.  
\item Tensia harvested another 7 cucumbers on Wednesday.  
\item Her neighbor Sarah graciously took 4 cucumbers to make pickles.  
\item Tensia herself ate 2 cucumbers during the week.  
\end{itemize}
How many cucumbers does she have left over?  \vfill

\item Brent's landlord charges \$15 per day for late rent.
\begin{enumerate} 
\item What will Brent's late fee be if is 6 days late paying his rent? 
\vfill  
\item If Brent got a bill showing \$195 in late fees, how many days late did he pay his rent? 
\vfill
\end{enumerate}

\newpage 

\item There are 2,624 students at a local university.

\begin{enumerate}
\item About $\frac{3}{4}$ of those students live on or within a mile of campus. How many students live on or within a mile of campus?   
\vfill 
\item The university wants to support 40 hours a week of onsite tutoring (in mathematics, writing, etc.) for each the 32 weeks that classes are in session.  It costs about \$18/hour to pay the tutors and staff administrators.  What is the total cost of tutoring? 
\vfill
\item The university is considering adding a tutoring fee to cover the cost of tutoring.  If they wanted to cover the total cost of tutoring, what would the cost per student be?
\vfill
\end{enumerate}


\item A truck hauling grass seed weighs 3,900 pounds when it is empty. Each bag of seed it carries weighs 4.2 pounds.  The \textbf{gross weight} of the truck is the total weight including the truck and the bags of seed. 

\hfill \emph{Story also appears in 2.1 \#1, 3.2 \#1, and 3.1 \#1}
\begin{enumerate}
\item How much does 1,300 bags of grass seed weigh? 
\vfill
\item What is the gross weight of the truck if it carries 1,300 bags of grass seed?
\vfill
\item You probably entered this calculation as $1300 \times 4.2 = + 3900=$.  What happens if you skip the middle = sign and enter $1300 \times 4.2 + 3900$ instead?
\vfill
\item What answer does your calculator give you if you enter $3900 + 4.2 \times 1300$ instead?
\vfill
\item What does part (d) tell you about which operation your calculator did first: the $+$ or the $\times$? 
\vfill
\end{enumerate}

\end{enumerate} % PAUSE

\newpage


\noindent \textbf{When you're done \ldots}

\begin{itemize}
\item [$\Box$] Check your solutions.  Still confused?  Work with a classmate, instructor, or tutor.
\item [$\Box$] Try the \textbf{Do you know} questions.  Not sure?  Read the textbook and try again.
\item [$\Box$] Make a list of key ideas and process to remember under \textbf{Don't forget!}
\item [$\Box$] Do the textbook exercises and check your answers. Not sure if you are close enough? Compare answers with a classmate or ask your instructor or tutor.  
\item [$\Box$] Getting the wrong answers or stuck?  Re-read the section and try again.   If you are still stuck, work with a classmate or go to your instructor's office hours or tutor hours.
\item [$\Box$] It is normal to find some parts of exercises difficult, but if most of them are a struggle, meet with your instructor or advisor about possible strategies or support services.
\end{itemize}





\bigskip

\noindent \textbf{Do you know \ldots}

\begin{enumerate}[(a)]
\item When to add, subtract, multiply, or divide numbers? \vfill
\item What is the difference between subtraction and negation? \vfill % pun intended :-)
\item How to add, subtract, negate, multiply, and divide on a calculator? \vfill
\item How multiplication is related to addition? \vfill
\item What the term ``per'' indicates? \vfill
\end{enumerate}

\noindent \textbf{Don't forget!}
\vfill \vfill \vfill




