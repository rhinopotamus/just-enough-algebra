%\section{Metric prefixes and scientific notation}

\begin{center}
\line(1,0){300} %\line(1,0){250}
\end{center}

\section*{Homework}

\noindent \textbf{Start by doing Practice exercises \#1-4 in the workbook.}

\bigskip

%SU should this be here:% SU -- check DoYouKnows because this is the first occurence of powers.

\noindent \textbf{Do you know \ldots}

\begin{itemize} 
\item How to calculate powers on your calculator?
\item What  million, billion, and trillion mean?  
\item Why metric prefixes are used?  
\item What common metric prefixes (mega, giga, kilo, centi, milli, micro, nano) mean? 

\emph{Ask your instructor which prefixes you need to remember, and whether any prefixes will be provided during the exam.} 
\item Why scientific notation is used?  
\item The standard format for scientific notation?  
\item What kinds of numbers have a positive order of magnitude, and which have a negative order of magnitude?
\item How to convert between decimal notation and scientific notation?  
\item How your calculator reports numbers in scientific notation, and what (might be) different when you're reporting that number?  
%\item How to enter numbers written in scientific notation into your calculator? 
\item The usual order of operations (PEMDAS) and how to use parentheses when you want a different order?
 \item[~] \textbf{If you're not sure, work the rest of exercises and then return to these questions.  Or, ask your instructor or a classmate for help.} 
\end{itemize}

\subsection*{Exercises}

\begin{enumerate} 
\setcounter{enumi}{4}

\item \begin{enumerate}
\item How many files at an average of 42.3 MB each can each gig (1 GB) of computer memory hold?
\item Tara's coworker Brandon has a much faster Internet connection on his computer at 1,500 kbps.  How long would it take Brandon to download 700 MB?  
\item At that rate, how much information could Brandon upload in 8 hours?  Express your answer in kilobytes (KB).
\end{enumerate}

\item \begin{enumerate}
\item Convert each of these amounts of time into an understandable unit of time: 1 million seconds, 1 billion seconds, 1 trillion seconds. 
\item Billy Bob wants to throw a party when he turns 1 billion seconds old. About how many years old will he be?
\item \emph{Bonus question:}  On what date were you or will you be 1 billion seconds old?  Don't forget leap years! \hfill \begin{footnotesize} Source:  Mathew Foss, North Hennepin Community College \end{footnotesize} % YES, only one t in Mathew.
\end{enumerate}  

\item  A proton has mass of about $1.67262 \times 10^{-27}$ kg, while an electron has mass of about $9.10938 \times 10^{-31}$ kg. 
\begin{enumerate}
\item Write out the mass of a proton and that of an electron in normal decimal notation.
\item Which is heavier (has greater mass)?
\item How many times heavier is it?  To calculate the answer take the mass of the heavier particle and divide it by the mass of the lighter particle.
\item How many protons would it take to weigh an ounce? Use 
$1 \text{ ounce} \approx 28.3 \text{ grams}$
and, as always, 1 kg = 1,000 grams.
\emph{Because $\times$ and $\div$ are at the same level in the order of operations, you should put parentheses around each number in scientific notation before dividing.}
\end{enumerate} % source:wiki.answers.com

\item How many servings are in 
\begin{enumerate}
\item A 2-liter bottle of a soft drink where the serving size is 250 mL?
\item A 750 mL bottle of wine where a serving size is 5 (fluid) ounces?  Use $1 \text{ quart} = 32 \text{ (fluid) ounces}$ and $1 \text{ liter} \approx 1.056 \text{ quarts}$.
\end{enumerate}

\item  Rayka weighs 140 pounds. She would like to approximate how many cells are in her body.  Use the following information: $1 \text{ cell} \approx 1 \times 10^{-15} \text{g}$, $1 \text{ kg} \approx \text{2.2 pounds} $, and, as always, 1 kg = 1,000 g.
\begin{enumerate}
\item How many cells are in Rayka's body?  Write your answer in scientific notation.
\item Rewrite your answer in the most appropriate unit:  millions ($10^6$), billions ($10^9$), trillions ($10^{12}$), quadrillions ($10^{15}$), or quintillions ($10^{18}$).
\end{enumerate}

\item  \textbf{Body Mass Index} (BMI) is one indicator of whether a person is a healthy weight.  BMI are between 18.5 and 24.9 are considered ``normal''.  Jared is 6'4" and weighs 200 pounds.  He would like to calculate his BMI from this guide:
$$\text{BMI} = \text{weight in kilograms} \div \text{height in meters} \wedge 2$$
\hfill \begin{footnotesize} Source:  Center for Disease Control and Prevention  \end{footnotesize}
%http://www.cdc.gov/healthyweight/assessing/bmi/adult_bmi/index.html#Interpreted
\begin{enumerate}
\item Check that Jared is around 1.93 meters tall and weighs around 90.91 kilograms.  Use $1 \text{ inch} \approx 2.54 \text{ cm}$ and $1 \text{ kilogram} \approx 2.2 \text{ pounds}$
\item Jared entered the following keystrokes on his calculator: $$90.91 \div 1.93 \wedge 2=$$
and got the answer $$\text{Jared's BMI } = 24.4060243\ldots$$ Is his BMI considered ``normal''?
\item Suppose Jared had rounded off his height to 1.9 meters and his weight to 91 kilograms.  Calculate his BMI by entering the following keystrokes your calculator:  $$91 \div 1.9 \wedge 2=$$
What do you get?  Round your answer to one decimal place.  Is Jared's BMI considered ``normal''?
\item What would you tell Jared?
\end{enumerate}






\end{enumerate}




%\item Here's one for Section 1.4  The artist Jeanne-Claude and Christo's Over the River installation will use 5.9 miles of fabric panels.  WiDTH??  How many yards of x wide is that?  In their 1973 Running fence installation they used approximately 200,00 sq meters of nylon.  In 1983 their Biscayne Bay installation in Florida topped that with 603,850 sq m of PINK polypropylene floating on the bay.
%source:  http://en.wikipedia.org/wiki/Christo_and_Jeanne-Claude

%\item Nanotechnology uses objects in the size range of 1 to 100 nanometers.  How small is that?  How many objects 1 nanometer would fit going across a human hair?  Recall a hair is around 0.00012 meters.







