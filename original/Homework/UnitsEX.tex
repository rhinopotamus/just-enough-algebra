%\section{Units}

 \begin{center}
\line(1,0){300} %\line(1,0){250}
\end{center}

\section*{Homework}

\noindent \textbf{Start by doing Practice exercises \#1-4 in the workbook.}

\bigskip

\noindent \textbf{Do you know \ldots}

 \begin{itemize}
\item How to convert from one unit of measurement to another?   
\item What a unit conversion fraction is?   
\item Why multiplying by a unit conversion fraction doesn't change the amount, just the units?   
\item How to connect repeated conversions into one calculation?   
\item Why if we convert an amount to a larger unit, we use a smaller number?  
\item How many seconds in a minute, minutes in an hour, hours in a day, days in a year, inches in a foot, feet in a mile, and other common conversions? 

\emph{Ask your instructor which common conversions you need to remember, and whether any conversion formulas will be provided during the exam.}   
\item How to convert between English and metric measurements? 

\emph{Again, ask your instructor which metric conversions you need to remember, and whether any conversion formulas will be provided during the exam.} 
 \item[~] \textbf{If you're not sure, work the rest of exercises and then return to these questions.  Or, ask your instructor or a classmate for help.} 
\end{itemize}

\subsection*{Exercises}

\begin{enumerate} 
\setcounter{enumi}{4}

\item In August 2008, United States swimmer Michael Phelps set the world record for the 200 meter individual medley swimming it in 1 minute, 54.80 seconds. 

 \hfill \begin{footnotesize} Source:  Wikipedia (World record progression 200 metres IM)  \end{footnotesize}
\begin{enumerate}
\item Convert Phelps' time into minutes.  
\item How fast did Phelps' swim, as measured in meters/min? 
\item Convert Phelps' speed to mph.   Use $1 \text{ mile} \approx {1,609} \text{ meters}$.
\end{enumerate}

In August 2012, Phelps improved his time and won Olympic gold, but failed to break the world record his teammate Ryan Lochte has set a year earlier of 1 minute, 54 seconds.
\begin{enumerate}
\item [(d)] Convert Lochte' time into minutes.  
\item [(e)]  How fast did Lochte' swim, as measured in meters/min?  
\item [(f)] Convert Lochte' speed to mph.
\end{enumerate}

\item \begin{enumerate}
\item The typical weight limit for a suitcase on flights within Africa is 20 kg.  How many pounds is that? Use $1 \text{ kilogram} \approx 2.2 \text{ pounds}$.
\item How many servings are in a 20 ounce package of cookies where a serving size is 3 cookies and each cookie weighs 11 grams?  Use $1 \text{ ounce} = 28.3 \text{ grams}$.
\item My corner convenience store sells a ``thirst quencher'' size of soft drink; it holds 64 (fluid) ounces.  If a can of soft drink is 12 (fluid) ounces, how many cans are in the ``thirst quencher''?
\end{enumerate}

\item \begin{enumerate}
\item The football coach wants everyone to sprint three-quarters of a mile, up and back on the field which is labeled in yards.  How many yards are in three-quarters of a mile?
\item The quilt pattern calls for 0.375 yards of calico fabric. How many feet is 0.375 yards?
\item The website said that basil plants should be 0.35 feet tall a month after germinating.  How many inches is 0.35 feet?
\end{enumerate}

\item Authorities are tracking down the source of a pollution spill on a nearby river.  They suspect that the local plant is inadvertently leaking waste water.  Last week they found 35 minutes of waste water flow on Monday, 1 hour and 11 minutes on Tuesday, 1/4 hour on Wednesday (that's 0.25 hours in decimal), none on Thursday, and then 98 minutes Friday.
\begin{enumerate}
\item Convert units as needed to complete the following table showing each time in minutes, each time in hours, and each time in hours and minutes (H:MM format). 

\emph{Hint:  15 minutes in H:MM format would be 0:15}
\begin{center}
\begin{tabular} {|l |c|c|c |c|c|} \hline
Day & Mon & Tue & Wed & Thu & Fri \\ \hline
Minutes & 35 & && 0 & 98 \\ \hline
Hours & \hspace{.5in}~ &\hspace{.5in}~ &.25&\hspace{.5in}~ &\hspace{.5in}~  \\ \hline
H:MM & & 1:11 & \hspace{.5in}~   & &  \\ \hline
\end{tabular}
\end{center}
\item Calculate the total waste water flow measured last week. 
\end{enumerate}

\item If your heart beats around 70 times a minute, how many times does it beat in a week?  A year?

\item \begin{enumerate}
\item Harold's Physics textbook says an object is thrown into the air at 36 feet per second.  To understand how fast that is, convert to mph.
\item Harold's History textbook mentions that in 1800 the city encompassed about 6,000 acres.  How many square miles is that?  Use $1 \text{ square mile} = 640 \text{ acres}$.
\item Harold's Economics textbooks lists the recent high price of crude oil at \$100 per barrel.  He'd like to know what that means in \$/gallon of gasoline.  It turns out that 1 barrel of crude oil produces about 19.4 gallons of gasoline. 
\end{enumerate}



%\item TEN Baseball player Joe Mauer signed a multi-year contract with the Minnesota Twins for an average of \$23 million per year.  (And that doesn't include the income he gets from endorsements.)
%\begin{enumerate}
%\item What does Mauer's salary come to in dollars per hour?  That means for every hour, waking or sleeping, all year long. 
%\item If Mauer were working a standard 40 hour work week for 50 weeks a year, what would his salary be, again in dollars per hour? \emph{Hint: that's a total of 2,000 hours} 
%\item In a standard 162 game season, averaging about 2 hours and 51 minutes per game, assuming Joe plays every minute of every game, what does his salary come to in dollars per game minute? \emph{Hint:  calculate the total number of minutes} 
%\end{enumerate}


%%\item WHERE? Every morning Jill goes for a 45-minute walk. 
%\begin{enumerate}
%\item Identify and name the variable.  Don't forget the units.
%\item  Which variable is independent?
%\item  If Jill walked 2.5 miles, how fast was she walking?  Don't forget to convert units as needed.
%\end{enumerate}   




\end{enumerate}




