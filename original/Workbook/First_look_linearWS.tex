%!TEX root =  A_WS.tex

\section{A first look at linear equations -- Practice exercises}

\begin{enumerate}

\item A truck hauling bags of grass seed pulls into a weigh station along the highway.  Trucks are weighed to determine the amount of highway tax.  This particular truck weighs 3,900 pounds when it is empty.  Each bag of grass seed it carries weighs 4.2 pounds. 
For example, a truck is carrying 1,000 bags of grass seed weighs 
$$\text{3,900 pounds } + \frac{4.2 \text{ pounds}}{\text{bag}} \ast \text{1,000  bags} = 3,900 + 4.2\times1,000=\text{8,100 pounds}$$ 
In official trucking lingo, we  say the \textbf{curb weight} of 3,900 pounds plus the \textbf{load weight} of 4,200 pounds results in a \textbf{gross weight} of 8,100 pounds.  So, now you know.  \hfill \emph{Story also appears in 3.1\#1 and 3.2 \#1}
\begin{enumerate}
\item Calculate the gross weight of the truck if it contains 2,000 bags of grass seed.   \vfill
\item Name the variables, including units, and  write an equation showing how the gross weight of the truck is a function of the number of bags of grass seed.  \vfill
\item Identify the slope and intercept, along with their units, and explain what each means in terms of the story. \vfill
\item The bags of grass seed are piled on wood \textbf{pallets} (sturdy platforms) to make them more stable for moving. How much does the truck weigh if it is carrying 12 pallets, where each pallet weighs 15 pounds and holds 96 bags of grass seed?    \vfill
\end{enumerate}

\newpage %%%%%%

\item The water in the local reservoir was 47 feet deep but there has been so little rain that the depth has fallen 18 inches a week over the past few months.  Officials are worried if dry conditions continue the reservoir will not have enough water to supply the town.  

\hfill \emph{Story also appears in 3.2 Exercises and 4.1 \#3}
\begin{enumerate}
\item Name the variables and write an equation relating them.  First convert 18 inches to feet. \vfill
\item Identify the slope and intercept, along with their units, and explain what each means in terms of the story. \vfill
\item Make a table of values showing the projected depth of the reservoir after 1 week, 5 weeks, 10 weeks, and 20 weeks if the current trend continues. \vfill
\item Draw a graph illustrating the function.
\begin{center}
\scalebox {.8} {\includegraphics [width = 6in] {GraphPaper.jpg}}
\end{center}
\end{enumerate}

\newpage %%%%%%

\item   I was short on cash so I got a  \textbf{line of credit} (short term loan) on my bank account, of which I spent \$2,189.57. That means my account balance is $-$\$2,189.57.  I will pay back the interest plus an extra \$250 each month.  When the loan is paid off,  I plan to continue to deposit \$250 per month to start saving. 
 
  \hfill \emph{Story also appears in 3.2 Exercises}
\begin{enumerate}
\item Write an equation showing my account balance, \$$B$, in $M$ months.  Ignore the interest. \vfill
\item Identify the slope and intercept, along with their units, and explain what each means in terms of the story.  \vfill
\item Make a table of values showing my account balance now, after 4 months, and at the end of a year.    \vfill \vfill
\item Draw a graph showing my account balance over this coming year.
\begin{center}
\scalebox {.8} {\includegraphics [width = 6in] {GraphPaper.jpg}}
\end{center}
\bigskip
\item About how many months will it take to pay off my line of credit?   \bigskip
\end{enumerate} 

\newpage %%%%%%

\item  A mug of coffee costs \$3.45 at Juan's favorite cafe, unless he buys their discount card for \$10 in which case each mug costs  \$2.90.
    \hfill \emph{Story also appears in 1.2 \#4 and 4.2 \#2}
\begin{enumerate}
\item Name the variables, including units. \vfill
\item Write an equation describing how the total cost depends on how many mugs of coffee Juan buys, assuming he does not buy the discount card. \vfill
\item Write an equation describing how the total cost depends on how many mugs of coffee Juan buys, if he buys the discount card. \vfill
\item How would the equation change if the cafe offers a new annual membership card that cost \$59.99 that entitles Juan to buy coffee for only \$1 per mug all year? \vfill
\end{enumerate} 

\end{enumerate}

\newpage


\noindent \textbf{When you're done \ldots}

\begin{itemize}
\item [$\Box$] Check your solutions.  Still confused?  Work with a classmate, instructor, or tutor.
\item [$\Box$] Try the \textbf{Do you know} questions.  Not sure?  Read the textbook and try again.
\item [$\Box$] Make a list of key ideas and process to remember under \textbf{Don't forget!}
\item [$\Box$] Do the textbook exercises and check your answers. Not sure if you are close enough? Compare answers with a classmate or ask your instructor or tutor.  
\item [$\Box$] Getting the wrong answers or stuck?  Re-read the section and try again.   If you are still stuck, work with a classmate or go to your instructor's office hours or tutor hours.
\item [$\Box$] It is normal to find some parts of exercises difficult, but if most of them are a struggle, meet with your instructor or advisor about possible strategies or support services.
\end{itemize}





\bigskip

\noindent \textbf{Do you know \ldots} % First_look_linear

\begin{enumerate} [(a)]
\item How to generalize from an example to find an equation? 
\item Where the dependent variable usually is in an equation? 
\item What the slope of a linear function means in the story and what it tells us about the graph? 
\item What the intercept of a linear function means in the story and what it tells us about the graph?  
\item The template for a linear equation? \emph{Ask your instructor if you need to remember the template or if it will be provided during the exam.} 
\item Where the slope and intercept appear in the template for a linear equation?  
\item What makes a function linear? 
\item How to plot negative numbers on a graph? 
\item What the graph of a linear function looks like? 
\end{enumerate}

\bigskip

\noindent \textbf{Don't forget!}
