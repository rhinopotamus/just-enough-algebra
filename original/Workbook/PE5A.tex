%!TEX root =  A_WS.tex

\section*{Practice Exam 5A}  
\markright{Practice Exam 5A}  
\addcontentsline{toc}{section}{Practice Exam 5A}

Relax.  You have done problems like these before.  Even if these problems look a bit different, just do what you can.  If you're not sure of something, please ask! You may use your calculator.  Please show all of your work and write down as many steps as you can.  Don't spend too much time on any one problem.  Do well.  And remember, ask me if you're not sure about something. \bigskip

\noindent \emph{As you work, make a ``don't forget'' list of any information you need to look up or ask about.} 

\noindent \hrulefill
\bigskip

\begin{enumerate} 

\item Leopard print hat. Originally 5 out of 1,000 women shopping at a major retail store even looked twice.  But that number grew and grew, by my estimate around 40\% a week, thanks to carefully placed ads in fashion magazines.
\begin{enumerate}
\item Write an equation illustrating the interest in leopard print hats using $W$ for the time (in weeks) and $L$ for the number of women interested in leopard print hats (women per thousand). 
\vfill
\item Make a table showing the number of women, per thousand female shoppers, who stop and look at the hat at the start, 1 week, 2 weeks, and 3 weeks after it hits the stores. 
\vfill
\item The leopard print hat is considered popular when more than 300 out of 1,000 women try it on.  According to the equation, when will the hat be considered popular?  Use successive approximation to find the answer to the nearest week and display your guesses in a table. 
\vfill
\vfill
\item The hat will be considered pass\'e when over 750 out of 1,000 women try it on.  I mean -- everyone's got one!  According to your equation, when will that happen?  Set up and solve an equation, again answering to the nearest week. 
\vfill
\vfill
\end{enumerate} 

\newpage

\item HeeChan bought a classic car in 2003 for investment purposes and has been watching the value increase over the years.  Based on the data, HeeChan came up with two possible equations
$$\textbf{Logistic:} \quad C= \frac{41,000}{1+4\ast.81^Y}$$

$$ \textbf{Saturation:} \quad C = 32,000-23,800\ast.85^Y$$

where $Y$ is the years since 2003 and \$$C$ is the value of the car.

\begin{enumerate}
\item How much did HeeChan pay for the car in 2003? 
\vfill
\item What does each equation predict for the value of the car in 2013?  In 2020?   
\vfill
\item What does each equation say will be the eventual value long term?  \emph{Hint:   if you are not sure try 100 years.}  
\vfill
\end{enumerate} 

\newpage

\item The number of geese in the Twin Cities metropolitan area increased from 480 in 1968 to 25,000 in 1994.  Although population is sometimes modeled with exponential models, there are many factors that might make an exponential model inappropriate, such as changes in migration, wetlands, and hunting.
\begin{enumerate}
\item Name the variables. 
\vfill
\item Write a linear equation modeling the goose population. 
\vfill
\vfill
\item Now write an exponential equation modeling the goose population. 
\vfill
\vfill

\newpage
\hspace{-.5in} \emph{The problem continues \ldots}

\item Compare the models projections for 1968, 1975, 1984, 1994, 2000, 2010, and 2020.  Summarize your findings in a table. 
\vfill
\vfill
\item Graph each function over the period from 1968 to 2020 on the same set of axes.   

\emph{Test-taking tip:  even if you have trouble with the equations, you should be able to plot the information given in the story and sketch in the appropriate shape curves.}

\bigskip
\begin{center}
\scalebox {.8} {\includegraphics [width = 6in] {GraphPaper.jpg}}
\end{center} 
\bigskip

\item Research indicates that the Twin Cities metropolitan area could support 60,000 geese.  Use your graph to estimate when that will happen. 
\vfill
\item The actual goose population in 2010 was around 50,000.  Which model was closer? 
\vfill
\end{enumerate}  

\newpage

 \item  One of the toxic radioactive elements produced by nuclear power plants is strontium-90. A large amount of strontium-90 was released in the nuclear accident at Chernobyl in the 1980's.  The clouds carried the strontium-90 great distances. The rain washed it down into the grass, which was eaten by cows. People then drank the milk from the cows.  Unfortunately, strontium-90 causes cancer. Strontium-90 is particularly dangerous because it has a half-life of approximately 28 years, which means that every 28 years half of the existing strontium-90 changes into a safe product; the other half remains strontium-90. Suppose that a person drank milk containing 100 milligrams of strontium-90.

\hfill \begin{footnotesize}Source:  ``Explorations in College Algebra,'' by Kime and Clark\end{footnotesize}
\begin{enumerate}
\item After 28 years, how many milligrams of strontium-90 remains in the person's body? After 56 years? 
\vfill
\item Find the annual percentage decrease of strontium-90. 
\vfill
\vfill
\item Name the variables and write an equation relating them. 
\vfill
\vfill
\item Suppose that any amount under 20 milligrams of strontium-90 is considered ``acceptable'' in humans. Will it have reached acceptable levels after 70 years? 
\vfill
\end{enumerate}  

%%% END

\end{enumerate}




