%!TEX root =  A_WS.tex

\section*{Practice Exam 2B}  
\markright{Practice Exam 2B}
\addcontentsline{toc}{section}{Practice Exam 2B}

\emph{Try taking this version of the practice exam under testing conditions:  no book, no notes, no classmate's help, no electronics (computer, cell phone, television). Give yourself one hour to work and wait until you have tried your best on all of the problems before checking any answers.}

\noindent \hrulefill

\begin{enumerate}

 \item The Sk\"arstroms want to dig a new well for water for their lake cabin.  The company charges \$900 to bring the equipment on site and draw the permit and then \$2 per foot to dig.
\begin{enumerate}
\item What would a 100 foot deep well cost? \vfill 
\item Name the variables and write an equation relating them. \vfill \vfill \vfill 
\item Make a table showing the total cost for a well 100, 250, or 400 feet deep. \vfill \vfill 
\end{enumerate} % MIGHT ALSO BE ON CHAPTER 3 PRACTICE EXAM  

\newpage %%

\item Xander grows tomatoes in his garden.  He's noticed that a typical plant yields 5 pounds of tomatoes.  He's been experimenting with the impact of liquid food on plant yield and estimates that each drop increases yield by 14\%.
\begin{enumerate}
\item Calculate the growth factor and write an equation showing how yield for each tomato plant depends on the number of drops of liquid food.  Use $Y$ for the yield (in pounds) and $D$ for the amount of liquid food (in drops).  \vfill  

\item Xander uses 10 drops of food on one of his tomato plans and uses all of the tomatoes from that plant to make salsa.  If each pound of tomatoes makes around a pint of salsa, how much salsa will Xander have (from that one plant)? \vfill 
\item Convert your answer into gallons.  Use $1 \text{ gallon} = 4 \text{ quarts}$ and $1 \text{ quart} = 2 \text{ pints}.$  Ol\'e! \vfill 
\newpage
\hspace{-.5in}  \emph{The problem continues \ldots}

\item Make a table showing Xander's projections for yield for each tomato plant  if he uses 0, 1, 2, 5, or 10 drops of liquid food.\vfill 

\item Graph the function.
\bigskip
\begin{center}
\scalebox {.9} {\includegraphics [width = 6in] {GraphPaper.jpg}}
\end{center}
\bigskip 
\vfill
\end{enumerate}  

\newpage %%

\item Skye and her sister Clover started a t-shirt printing company.  To produce a particular t-shirt it costs  \$350 in materials and labor to set up a silkscreen and then \$7.50 for each shirt made to cover materials and printing.  The average cost per t-shirt \$$C$ is a function of $N$, the number of t-shirts printed.  The equation for this function is $$C = \frac{350+7.50N}{N}$$

\begin{enumerate}
\item Evaluate this formula when $N=50$ and explain what the value of $C$ you get means in the story. \vfill 
\item Make a table showing the average cost per t-shirt if Skye and Clover make 1, 20, 50, 100, or 300 t-shirts. \vfill 
\item  Approximately how many t-shirts would they need to make to keep the average cost per shirt under \$10? 
Use successive approximation and display your guesses in a table. \vfill  \vfill
\end{enumerate}

\newpage %%
\hspace{-.5in}  \emph{The problem continues \ldots}

Skye designs the shirts and runs the press.   Clover is the brains behind sales.  She would like to price the shirts at \$12.95 each.  The sisters will make a profit of \$$P$ where $$P = 5.45N-350$$ 
\begin{enumerate}
\item [(d)] This is a linear equation.  What is the slope, what are its units, and what does it mean in the story?  \vfill 
\item [(e)] What is the intercept, what are its units, and what does it mean in the story?  \vfill 
\item [(f)] How many t-shirts do the sisters need to sell to 
make \$1,000 profit?
Use successive approximation and display your guesses in a table. \vfill \vfill
\end{enumerate} 

\newpage %%

 \item  \begin{enumerate} 
\item Kotoyo's uncle won \$100,000 on a game show.  If he invests it in a fund that is expected to earn 5.7\% interest compounded monthly, how much will he have after 5 years? Use the \textsc{Compound Interest Formula}.  \vfill 
\item Kotoyo's grandmother has been contributing \$150 a month into a college fund for Kotoyo for the past 8 years.  The account pays 4\% interest compounded monthly.  How much is in the account now? Use the \textsc{Future Value Annuity Formula}. \vfill  
\item Kotoyo owes \$8,742 on her credit card.  They charge her 16\% interest compounded monthly.  What would her monthly payment be if she wants to pay it off in 5 years? Use the \textsc{Loan Payment Formula}.  \vfill 
\item What is the equivalent annual percentage rate (APR) of Kotoyo's credit card? Use the \textsc{Equivalent APR Formula}.  \emph{Don't forget to report the percentage.} \vfill 
\end{enumerate} 

%%%% END

\end{enumerate}

