\appendix

\chapter{More about \ldots}

You have been using and learning mathematics all of your life.  Sometimes in school.  Other times at home or on the job.  But if you are like most people, you have likely forgotten many details over the years.  Moreover, there are probably concepts that you never really did understand.  

Mathematics that you use everyday is probably very familiar to you.  Perhaps you have taken another mathematics course in college already, or had other classes where you have reviewed some mathematics.  Or maybe you were reminded of some things when you helped a child with her homework.  Only one thing is certain -- each of us remembers, knows, and understands a different collection of mathematics.  

Where should a mathematics book begin, then?  What mathematics can we assume most students already know?  What can we review quickly?  What is new to most students?  In this course we take the approach that reviewing at the time we need a piece of information works well for most people.  (That philosophy is known as ``just in time'' instruction.) Throughout this text we try not to assume too much and try to include all of the mathematics you need to know.  

There are times, however, when we likely review a topic too quickly for some students.  If you find yourself in that situation, consider this Appendix your safety net.  Hopefully you will find here enough additional instruction and practice with the core prerequisite knowledge for those times.  If not, be sure to ask your instructor for advice.

A reminder that since everything you need to know is in the main text, if the course is going well you might find you never use this Appendix.  But, it's here if you need it.