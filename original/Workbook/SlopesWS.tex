%!TEX root =  A_WS.tex

\section{Slopes -- Practice exercises}

\begin{enumerate}

\item For his Oscars' party, Harland had 70 chicken wings delivered for \$51.25.  For his Super Bowl bash, Harland had 125 chicken wings delivered for \$83.70. In each case, the total cost includes the cost per wing and the fixed delivery charge. 
%slope = \$3.89/pound??? 59c/wing or 7.08/dozen
%intercept = \$17.50?
\begin{enumerate}
\item Find the slope, including units, and explain what it means in the story.\vfill
\item Find the intercept, including units, and explain what it means in the story.\vfill
\item Name the variables and write an equation for the function.\vfill
\item How many wings could Harland order for \$100?  Solve your equation.
\vfill \vfill
\item Graph and check.
\begin{center}
\scalebox {.8} {\includegraphics [width = 6in] {GraphPaper.jpg}}
\end{center}
\bigskip 
\end{enumerate}

\newpage %%%%%%

\item Jana is making belts out of leather strips and a metal clasp.  A short belt (as shown) is 24.5 inches long and includes 7 leather strips.  A long belt (not shown) is 37.3 inches long and includes 11 leather strips.  Each belt includes one metal clasp that is part of the total length.  All belts use the same length clasp.
\begin{center}
\scalebox{.5} {\includegraphics{leatherbelt.pdf} }
\end{center}
\begin{enumerate}
\item Name the variables, including units. \vfill
\item How long is each leather strip? \vfill
\item How long is the metal clasp?  \vfill
\item Write an equation relating the variables. \vfill
\item Solve your equation to find the number of leather strips in an extra long belt that is 43.7 inches long. \vfill \vfill
\end{enumerate}

\newpage %%%%%%

\item The local ski resort is trying to set the price for season passes.  They know from past experience that they will sell around \text{14,000} passes if the season ticket price is \$380.  If the price is \$400, they will sell fewer, perhaps only \text{11,000} passes.  You can assume this decrease in demand is linear.
\begin{enumerate}
\item For every dollar increase in price, how many fewer people purchase season passes? \vfill
\item Find the intercept.  Explain why this number does not make sense in the problem. \vfill
\item Write an equation for the function, using $T$ for the ticket price, in dollars, and $D$ for the demand (number of tickets sold). \vfill
\item How many season passes will they sell if the price is reduced to \$355? \vfill
\item The amount of \textbf{revenue} (money they take in) depends both on the ticket price and the number of tickets sold.  The equation is $R = TD$, where $R$ is the revenue, in dollars.  Calculate the revenue when ticket prices are \$355, \$380, and \$400.  \emph{That means multiply the ticket price $T$ times the number of tickets sold $D$ in each case listed.}  
\vfill 
\item Of these three prices, which yields the most revenue? \vfill 
\end{enumerate}

\newpage %%%%%%

\item Boy, am I out of shape.  Right now I can only press about 15 pounds. (\textbf{Press} means lift weight off my chest.  Literally.)  My trainer says I should be able to press 50 pounds by the end of 10 weeks of serious lifting. I plan to increase the weight I press by a fixed amount each week.
\begin{enumerate}
\item Name the variables and write an equation for my trainer's projection. 
  
 \emph{Hint:  you know the intercept.} \vfill
 \item Make a table showing my trainer's projection for after 0, 5, 10, 15, and 20 weeks.
\vfill
\item Years ago I could press 90 pounds. At this rate, when will I be able to press at least 90 pounds again?  Set up and solve an inequality. \vfill

\newpage %%%%%%
~\hspace{-.5in} \emph{The problem continues \ldots}

\item Draw a graph illustrating the function.
\begin{center}
\scalebox {.8} {\includegraphics [width = 6in] {GraphPaper.jpg}}
\end{center}
\bigskip 
\bigskip 
\item I am skeptical.  I do not think I will be able to press 50 pounds by the end of 10 weeks.  If I revise my equation, will the new slope be larger or smaller?  

 \emph{Hint:  try sketching in a possible revised line on your graph assuming that after 10 weeks I will press much less than 50 pounds.} \vfill
\item Will my revised projections mean I will reach that 90-pound goal sooner or later?  Explain.  \emph{Hint: extend your graph.} \vfill
\end{enumerate}

\end{enumerate}

\newpage


\noindent \textbf{When you're done \ldots}

\begin{itemize}
\item [$\Box$] Check your solutions.  Still confused?  Work with a classmate, instructor, or tutor.
\item [$\Box$] Try the \textbf{Do you know} questions.  Not sure?  Read the textbook and try again.
\item [$\Box$] Make a list of key ideas and process to remember under \textbf{Don't forget!}
\item [$\Box$] Do the textbook exercises and check your answers. Not sure if you are close enough? Compare answers with a classmate or ask your instructor or tutor.  
\item [$\Box$] Getting the wrong answers or stuck?  Re-read the section and try again.   If you are still stuck, work with a classmate or go to your instructor's office hours or tutor hours.
\item [$\Box$] It is normal to find some parts of exercises difficult, but if most of them are a struggle, meet with your instructor or advisor about possible strategies or support services.
\end{itemize}





\bigskip

\noindent \textbf{Do you know \ldots} % Slopes

\begin{enumerate} [(a)]
\item Which types of situations are linear? 
\item What the slope of a linear function means in the story and what it tells us about the graph? 
\item How to calculate the slope between two points? 
\item What is means if the slope is negative? 
\item How to find the equation of a line through two points? 
\item How to find a linear function given two examples in a story? 
\item If both the slope and intercept are unknown, which is easier to calculate first? 
\end{enumerate}

\bigskip

\noindent \textbf{Don't forget!}


