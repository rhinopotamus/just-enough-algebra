\section{Prelude: Scientific Notation}

Tara is working on a big project at work.  She wants to back up her files to her online drop box. The site says she has 72 GB of memory remaining.  Tara has about 200 files at an average of 42.3 MB each that she would like to upload.  Will she have room?

To answer Tara's question we need to know that GB is short for ``gigabyte'' and MB is short for ``megabyte.''  A \textbf{byte} is a very small unit of computer memory storage space just enough for about one letter.  You may have heard the word ``mega'' used to mean ``really big.''  There's a reason for that.  \textbf{Mega} is short for 1 million.  That's pretty big.  But \textbf{giga} stands for 1 billion, so that's even bigger. So, really
\begin{center}
\begin{tabular} {lclcr} 
\textbf{megabyte} &$=$&1 \textbf{ million bytes} &$=$&$ \text{1,000,000 bytes}$\\
\textbf{gigabyte} &$=$& 1 \textbf{ billion bytes} &$=$& $\text{1,000,000,000 bytes}$\\ 
\end{tabular}
\end{center}

What all this means is Tara has
$$72\text{ GB} = 72 \text{ billion bytes} = \text{72,000,000,000 bytes}$$
of memory remaining.
She would like to save 200 files at 42.3 MB each which comes to
$$200 \times 42.3 = \text{8,460 MB}$$
which is really 
$$ \text{8,460 MB} = \text{8,460 million bytes} = \text{8,460,000,000} \text{ bytes}$$
$$= 8.46 \text{ billion bytes} = 8.46 \text{ GB }$$
So Tara wants to store less than 9 GB of information and she has 72 GB remaining.  She has plenty of room.  Save away!

Really large numbers, like \text{8,460,000,000}, are awkward to read and awkward to work with.  Words like million and billion or metric prefixes (words like mega and giga)  allow us to rewrite these large numbers in a way that's much easier both to read and to work with.  There's another option that's used often in the sciences (and by your calculator).  To explain it we need to understand powers of 10.

 Perhaps you know what happens when we multiply a number by 10, like
$ 5 \times 10 = 50$ or, more appropriate to our example, $$ 8.46 \times 10 = 84.6$$
The effect of multiplying by 10 is to move the decimal point one place to the right.
When we multiply by \text{1,000} we get
$ 5 \times \text{1,000} = \text{5,000} $ or, for our example, $$8.46 \times \text{1,000} = \text{8,460}$$
The effect of multiplying by \text{1,000} is to move the decimal point three places to the right.
The connection is that $$10^3=10 \times 10 \times 10  =\text{1,000}$$
Each $\times 10$ has the effect of moving the decimal point one place to the right so $\times \text{1,000}$ has the same effect as multiplying by 10 three times, so the decimal point moves three places to the right.
That means 
\begin{eqnarray*}
\text{8,460,000,000} & = & 8.46 \underbrace{\hbox{$\times 10 \times 10  \times 10  \times 10  \times 10  \times 10  \times 10  \times 10  \times 10 $}}_{\hbox{\begin{tiny} 9 times \end{tiny}}}  \\  %HSPACE
& = & 8.46 \ast 10^ 9 \\  
\end{eqnarray*} 
\vspace{-.5in} %VSPACE

This shorthand is called \textbf{scientific notation}.  The base is always 10.  The exponent is always a whole number.  The number out front, like 8.46 in our example, is always between 1 and 10, which means there's exactly one digit before the decimal point (and any others must come afterwards).  It is customary to use $\times$ instead of $\ast$ in scientific notation, so we should write
$$8.46\times10^9$$

As another example, we saw earlier that $$\text{5,000} = 5 \times \text{1,000} = 5 \times 10^3$$
You can check that  $$5 \times 10 \wedge 3 = 5000$$
 
Back to our large number.  Enter $$8.46 \times 10 \wedge 9=$$
What do you see?  Some calculators correctly show $\text{8,460,000,000}$ while other calculators report the number back in scientific notation, which is not particularly useful. (Sigh.)  

Let's try a number so big that (nearly) every calculator will switch to scientific notation.  Enter $$2.7\times 10 \wedge 30=$$
Look carefully at the screen.  Your calculator might display something like 
$$\boxed{~2.70000000 \quad \text{\begin{footnotesize}E \end{footnotesize}}~30~} \text{ \quad or \quad } \boxed{~2.70000000 \quad _{ \times \text{\begin{tiny}10 \end{tiny}}}~30~}$$ 
Whatever shorthand your calculator uses, you should write $$2.7 \times 10^{30}$$

Interested in what that number is in our usual decimal notation?  It's
$$2, \underbrace{ \hbox{$\text{700,000,000,000,000,000,000,000,000,000} $}}_{\hbox{\begin{tiny} decimal point moves 30 places \end{tiny}}}$$

Enough of that. Poor Tara is pulling her hair out over this project.  Well, not literally, but she is quite frustrated over how slowly the project is going.  She wonders: how thick is a human hair?  

Turns out that a typical human hair is about $.00012$ meters across.  Very small numbers are also awkward to read and awkward to work with.  In this section, we write  $.000~12$ where the strange-looking space is to help you read the number.  
 
We can also describe really small numbers using scientific notation.  Perhaps you know what happens when we divide a number by 10, like
$ 50 \div 10 = 5$ or, more appropriate to our example, $$ 1.2 \div 10 = .12$$
The effect of dividing by 10 is to move the decimal point one place to the left.
If we divide by \text{1,000,000} instead, we get
$$ 1.2 \div \text{1,000,000} = .000~001~2 $$ 
The connection is that $$ \text{1,000,000} = 10 \wedge 6  $$
and so dividing by \text{1,000,000} moves the decimal point six places to the left.  Notice that we have to introduce zeros as placeholders.

The width of a hair was .00012 meters.  To get that number from 1.2, we need to move the decimal point 4 places to the left.  $$1.2 \div 10^4 = 1.2 \div \text{10,000} = .000~12$$
The shorthand for dividing by a power is to use negative exponents.  For example
$$ \div 10^4 = \times 10^{-4}$$
It has nothing to do with negative numbers.  It's just a shorthand.
The point of this calculation was that $$.00012= 1.2 \ast10^{-4}$$
Use your calculator to check!

Once again we have scientific notation.  The base is still 10.  The exponent is still a whole number, although now it's negative.  The number out front, like 1.2 in our example, is still between 1 and 10, which means there's exactly one digit before the decimal point (and any others must come afterwards).  As before, we'll write $\times$ instead of $\ast$ to get: $$1.2 \times 10^{-4}$$


When you see a number written in scientific notation, the power of 10 tells you a lot.  For example, we saw that $8.46 \times 10^ 9 = \text{8,460,000,000}$ and $1.2 \times10^{-4} = .000~12 $.  A positive power of 10 says you have a big number, and a negative power of 10 says you're dealing with a very small number. 
\bigskip

\noindent \textbf{Do you know \ldots}

\begin{itemize}
\item What million, billion, and trillion mean?  
\item Why scientific notation is used?  
\item The standard format for scientific notation?  
\item That a positive exponent corresponds to a big number and a negative exponent corresponds to a tiny number?
\item How to convert from scientific notation to decimal?  
\item How your calculator reports numbers in scientific notation, and what (might be) different when you're reporting that number?  
 \item[~] \textbf{If you're not sure, work the rest of exercises and then return to these questions.  Or, ask your instructor or a classmate for help.} 
\end{itemize}

\subsection*{Exercises}

\begin{enumerate} 
\setcounter{enumi}{4}

\item \emph{Story also appears in 1.5 \#6}
\begin{enumerate}
\item Convert 1 million seconds into an understandable unit of time. 
\item Billy Bob wants to throw a party when he turns 1 billion seconds old. About how many years old will he be?
\item \emph{Bonus question:}  On what date were you or will you be 1 billion seconds old?  Don't forget leap years! \hfill \begin{footnotesize} Source:  Mathew Foss, North Hennepin Community College \end{footnotesize} % YES, only one t in Mathew.
\end{enumerate}  

\item \emph{Story also appears in 1.5 \#3} 
\begin{enumerate}
\item The planet Jupiter weighs approximately $\mathbf{1.9 \times 10^{27}}$ \textbf{kilograms}. Write out this number (with all the zeros).
\item The planet Mars weighs approximately $\mathbf{6.4 \times 10^{23}}$ \textbf{kilograms}. Write out this number (with all the zeros).
\item Which planet weighs more:  Jupiter or Mars?  Explain. 
\end{enumerate}

\item \begin{enumerate}
\item The SARS CoV-2 virus is approximately 125 nanometers wide which is 
$\mathbf{125 \times 10^{-9}}$ \textbf{meters} wide. Write out this number (with all the zeros).
\item The N95 mask prevents particles down to 0.3 microns which is 
$\mathbf{3 \times 10^{-6}}$ \textbf{meters} wide but not smaller. Write out this number (with all the zeros).
\item Can the N95 mask prevent the SARS CoV-2 virus?  Explain.
\end{enumerate}

\item  Rayka would like to approximate how many cells are in her body.  Use the following information: Rayka weighs 140 pounds, $1 \text{ gram} \approx 10^{15} \text{ cells}$ and $\text{1,000 grams} \approx \text{2.2 pounds}$.
\hfill \emph{Story also appears in 1.5 \#9}
\begin{enumerate}
\item How many cells are in Rayka's body?  Hint: this is a unit conversion question asking you to convert 140 pounds to cells.  Write your answer in scientific notation.
\item Rewrite your answer in the most appropriate unit:  millions ($10^6$), billions ($10^9$), trillions ($10^{12}$), quadrillions ($10^{15}$), or quintillions ($10^{18}$).
\end{enumerate}

\end{enumerate}

\bigskip

\noindent \textbf{When you're done \ldots}

\begin{itemize}
\item Don't forget to check your answers with those in the back of the textbook. 
\item Not sure if your answers are close enough? Compare with a classmate or ask the instructor.  
\item Getting the wrong answers or stuck on a problem?  Re-read the section and try the problem again.   If you're still stuck, work with a classmate or go to your instructor's office hours.
\item It's normal to find some parts of some problems difficult, but if all the problems are giving you grief, be sure to talk with your instructor or advisor about it.  They might be able to suggest strategies or support services that can help you succeed.
\item Make a list of key ideas or processes to remember from the section.  The ``Do you know?'' questions can be a good starting point.
\end{itemize}


