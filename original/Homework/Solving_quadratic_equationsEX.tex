%\section{Solving quadratic equations}

\begin{center}
\line(1,0){300} %\line(1,0){250}
\end{center}

\section*{Homework}

\noindent \textbf{Start by doing Practice exercises \#1-4 in the workbook.}

\bigskip

\noindent \textbf{Do you know \ldots}

\begin{itemize}
\item What a ``quadratic'' function is? 
\item How to solve a quadratic equation? 
\item When we use the \textsc{Quadratic Formula}?   \emph{Ask your instructor if you need to remember the \textsc{Quadratic Formula} or if it will be provided during the exam.}
\item How to solve a quadratic equation when the function is not set equal to zero? 
\item How to identify the values of $a, b, c$ in the formula? 
\item How to evaluate the formula (using your calculator)?   \item Why there are (usually) two solutions to a quadratic equation? 
\item How to decide which solution(s) from the \textsc{Quadratic Formula} are correct? 
\item What the graph of a quadratic function looks like? 
\item The value for the independent variable to find  the highest (or lowest) value of a quadratic function? 
 \item[~] \textbf{If you're not sure, work the rest of exercises and then return to these questions.  Or, ask your instructor or a classmate for help.}
\end{itemize}

\subsection*{Exercises}

\begin{enumerate} 
\setcounter{enumi}{4}

\item Claude is an excellent juggler.  Remember that the height $H$ feet of Claude's beanbag $T$ seconds after he throws it in the air is described by the equation $H = 3+15T-16T^2$. Answer each of the following question by the suggested method and then look back at the graph from earlier to make sure your answers make sense.
\begin{enumerate}
\item Use the \textsc{Quadratic Formula} to find when is the bean bag is 5 feet above ground?  Why do both answers make sense in the story?
\item When is the beanbag 8 feet above ground?  Try to use the \textsc{Quadratic Formula} to find the answer.  What happens?  Explain why it makes sense in the story that you can't solve this quadratic equation.
\item Claude decided that the beanbag was too high in the air, so he modified his throw slightly.  Now the height is given by $H = 3+14T-16T^2$.  What is the maximum height the beanbag will reach now?  \emph{Hint:  what number can you evaluate at?}
\end{enumerate}

\item The stopping distance for Jeff 's Cadillac Escalade is given by  $$D=.04S^2+1.47S$$ where $S$ is the speed of the car (in miles per hour) and $D$ is the stopping distance (in feet). Jeff took 183 feet to stop.  How fast was he going? 

\hfill \emph{Story also appears in Section 2.3}
\begin{enumerate}
\item Use successive approximation to estimate the answer to the nearest miles per hour.  Display your work in a table.
\item Show how to use the \textsc{Quadratic Formula} to solve the equation.
\end{enumerate}

\item A company produces backpacks.  The more they make, the less it costs for each one. The cost per backpack is given by the equation $$C = .01B^2 -1.2B + 50$$ where $C=$ cost per backpack (\$ per backpack) and $B=$ number of backpacks.

\hfill \emph{Story appears in 1.3 Exercises}
\begin{enumerate}
\item How many backpacks do they need to produce in order to hold costs to \$20/backpack?  Set up and solve a quadratic equation to find the answer.
\item Make a table of values and draw a graph of the function. Does your answer agree with your table and graph?  
\item What is the minimum price per backpack?  \emph{Hint:  evaluate at $T= \frac{-b}{2a}$.}
\end{enumerate}

\item Mrs.\ Weber's cooking class came up with the equation $$M = 1.2F^2+4F+7$$ to approximate the grilling time of a piece of fish depending on its thickness.  Here $M$ is the number of minutes to grill the fish and $F$ is the thickness of the fish in inches.  

\hfill \emph{Story appears in 1.1 and 2.3 Exercises}
\begin{enumerate}
\item If we want to make sure the fish will cook in under 20 minutes, what thickness steak can we have? Set up and solve a quadratic equation to find the answer.
\item Repeat for 10 minutes. 
\end{enumerate}

\item A company who makes electronics was doing great business in 1996, but sales quickly slid after 2000.  Their sales $M$ in millions of \$ $Y$ years from 1996 is given by the following equation $$M = 104.4+11.5Y-1.4Y^2$$
\hfill \emph{Story appears in 2.4 Exercises}
\begin{enumerate}
\item The company decided to declare bankruptcy when sales fell below \$20 billion.  In what year was that?    Show how to solve using the \textsc{Quadratic Formula.}
\item An analyst had suggested that they close down shop earlier, once sales were below \$50 billion.  In what year did sales fall that low? Show how to solve using the \textsc{Quadratic Formula.}
\item What year did sales \textbf{peak} (reach their highest value)?  
\end{enumerate} 

\item A kangaroo hops up in the air (and out) from a 10 foot cliff.  Her height above the ground, $K$ feet, after $T$ seconds is given by the equation $$K = 10 + 5.2T - 4.88T^2$$ 
\begin{enumerate}
\item Calculate the missing values in the table. 
\begin{center}
\begin{tabular} {|l|c|c|c|c|c|c|} \hline
T & 0 & .3 & .6 & .9 & 1.2 & 1.5 \\ \hline
K& 10 & 11.1208 & 11.3632  &  \hspace{.4in}    &  \hspace{.4in}   &6.82  \\ \hline
\end{tabular}
\end{center}
\item According to this equation, how high up in the air does the kangaroo get? 
Choose the appropriate value to plug into the equation.
\item When does the kangaroo land on the ground? 
Set up and solve an equation.
\item If she jumps up, but not out, when will she land on the cliff itself again, assuming the same equation holds?
\end{enumerate}

\end{enumerate}

